\section{Names}

In T1, functions, types and local variables have names. A name is a
sequence of characters (unicode code points). In general, any sequence
of code points is usable for any purpose. However, names that are
encountered within source code are subject to some processing which
modifies their interpretation and restricts their syntax.

The lexer parses a name as a token that contains only printable non-space
ASCII characters, excluding a few ``forbidden'' characters (parentheses,
braces...). Such a name may be ``qualified'' or ``unqualified'':
\begin{itemize}

    \item A \emph{qualified name} contains a single instance of the
    sequence ``\verb|::|''. The part before the sequence is the
    \emph{namespace}, and the part after is the \emph{raw
    name}.\footnote{In general a name shall not contain more than one
    instance of ``\texttt{\textbf{::}}''. If it does, then all
    operations that split the name into a namespace and a ``raw'' name
    use the first (leftmost) occurrence of ``\texttt{\textbf{::}}'' as
    splitting point.}

    \item An \emph{unqualified name} does not contain the sequence
    ``\verb|::|''. Thus, raw names (obtained from removing the namespace
    from a qualified name) are unqualified.

\end{itemize}

The general model of source code interpretation is that qualified names
designate a specific entity, and unqualified names are interpreted
depending only on the syntactic context. For instance, under normal
conditions, the interpreter reads the next token and expects it to be
either a numerical constant, a literal string, or a function name. In
this last case:
\begin{itemize}

    \item If the name is qualified, then this designates exactly that
    function.

    \item If the name is not qualified, then it will be matched against
    the following, in due order:
    \begin{itemize}

        \item If the name matches an accessor for a locally allocated
        variable or instance, then the name is interpreted as an
        invocation of that accessor.

        \item If the name is part of one of the currently defined aliases
        (imports), then it designates the function that the alias maps
        to.

        \item Otherwise, the name is considered to use the currently
        active namespace.

    \end{itemize}

\end{itemize}

Namespaces are used to keep track of defined functions and types, and to
avoid spurious collisions. Normal source code should contain very few
qualified names:
\begin{itemize}

    \item At any point in the source code, there is an active namespace
    in which new functions and types are defined.

    \item Access to names in other namespaces is normally done through
    aliases and import lists.

    \item The developers are supposed to have full knowledge of ``their''
    namespaces, and thus avoid internal collisions.

\end{itemize}

There are no visibility rules, i.e. public and non-public functions and
types. Every such element \emph{can} be accessed by using a qualified
name. However, good software engineering practice is to refrain from
doing so in the general case. Functions and types can be added to the
\emph{export list} of the currently active namespace: such an export
list can be imported from other namespaces with an ``\verb|import|''
clause, which locally defines corresponding aliases for the exported
names. In that sense, ``public'' functions can be defined by making them
part of the export list of their namespace, which documents the intent
of making them callable from other namespaces.

More details on import lists and aliases are given in
section~\ref{syntax:namespace}.

All namespaces that start with ``\verb|std|'' are reserved for the T1
implementation. Notably, the export list for ``\verb|std|'' itself is
automatically imported, and it defines aliases for all the core
syntactic constructions and types. Source code can, at any time, clear
the current list of aliases, including those from the ``\verb|std|''
export list.
