\section{Types}\label{types}

All T1 values are \emph{references} to \emph{instances}. This includes
the basic types (booleans, small integers...). Formally, a value for the
integer ``5'' is considered to be a reference to an immutable object
instance that represents that integer; such instances are virtual and
cannot be really created in memory. This definition allows us to define
the T1 type system without making special cases for such basic types. In
practice, there are some restrictions in compiled code that avoid type
punning that would be expensive to implement.

\subsection{Sub-Typing}

\emph{Sub-typing} is a mechanism which incarnates promises of
functionality. When type \verb|bar| is a sub-type of \verb|foo|, then it
means that whenever a function expects as argument a reference to an
object of type \verb|foo|, it may receive instead a reference to an
object of type \verb|bar|, and things ``should work''. As we shall see
later on, the only thing that can be made with values is to call
functions on them, and function calls are dynamically mapped based on
the runtime types of their arguments; thus, making \verb|bar| a sub-type
of \verb|foo| means that for every function that takes as input a
\verb|foo|, a function of the same name that accepts a \verb|bar| is
defined.

Whether such promises are fulfilled or not does not impact sub-typing.
During interpretation, an unfulfilled promise will trigger a runtime
error at the time the function is called. The compiler statically checks
that such a situation does not occur; in that sense, the compiler does
not check that there are methods for \verb|bar| that correspond to all
methods for \verb|foo|, only that there are such methods for all calls
that can actually occur in the compiled code.

Sub-typing has the following rules:
\begin{itemize}

    \item Every type is considered to be a sub-type of itself. A
    \emph{strict sub-type} of type ``\verb|T|'' is a sub-type of
    ``\verb|T|'' that is not ``\verb|T|'' itself.

    \item All types are sub-types of ``\verb|std::object|''.

    \item Strict sub-typing is an acyclic graph. A given type is
    a sub-type of itself, but shall not be a sub-type of any
    strict sub-type of itself.

    \item Sub-typing rules can be added to any type at any time,
    provided that they don't create cycles. Basic types are an
    exception, in that they cannot be sub-typed, or made strict
    sub-types of any other types except ``\verb|std::object|''. Note
    that sub-typing can be added even on types for which instances have
    already been created.

\end{itemize}

Sub-typing is distinct from both embedding and extensions, which will be
covered later on.

If \verb|A| is a sub-type of \verb|B|, then \verb|B| is a \emph{super-type}
of \verb|A|. A type can have several direct sub-types and several direct
super-types. Sub-types and super-types are not ordered.

\begin{rationale}
Sub-typing is somewhat similar to interfaces in Java and C\#: a
mechanism to tag types, and promise existence of methods, without
actually implementing them. However, in both Java and C\#, such promises
are checked by the compiler: a class that declares that it implements a
given interface, but fails to provide the relevant methods, will be
rejected at compilation. This does not occur in T1, which concentrates
on actual usage: a compilation error is triggered not by failure of
providing a method that matches a given sub-typing relationship, but
by trying to call such a method.
\end{rationale}

\subsection{Basic Types}

\emph{Basic types} are the following:
\begin{itemize}

    \item \verb|std::object| is the root of the sub-typing graph.
    Instances of \verb|std::object| cannot be created. The basic
    equality and inequality functions (``\verb|=|'' and ``\verb|<>|'')
    are defined on \verb|std::object| and implement (in)equality of
    references. Due to the way function lookups are performed, this
    behaviour is inherited by all types, unless explicitly overridden.

    \item \verb|std::bool| is the boolean type. The two possible values
    are ``\verb|true|'' and ``\verb|false|''.

    \item \verb|std::int| is the default integer type. It has a range
    which depends on the current architecture and implementation, but
    which is large enough to serve as index value in arrays. It is
    signed: its range is $-m$ to $m-1$ for a given integer $m = 2^\ell$.
    Operations on \verb|std::int| are checked; any overflow condition
    triggers a runtime error.

    \item \verb|std::u8|, \verb|std::u16|, \verb|std::u32| and
    \verb|std::u64| implement unsigned integers modulo $2^8$, $2^{16}$,
    $2^{32}$ and $2^{64}$, respectively. Since they are modular integers,
    they have ``wrap-around'' semantics, and never overflow.

    \item \verb|std::i8|, \verb|std::i16|, \verb|std::i32| and
    \verb|std::i64| are the signed types corresponding to the unsigned
    modular types. For instance, \verb|std::i8| is for integers in the
    $-128$ to $127$ range. They implement wrap-around semantics, just
    like unsigned types; like Java and unlike C, computations on signed
    types that exceed the range yield well-defined results.

\end{itemize}

\emph{There are no automatic conversions.} Contrary to C, no value can
be used as a boolean, except the values of type \verb|std::bool|.
Arithmetic operations such as addition cannot involve distinct types;
e.g. you cannot add a \verb|u8| to a \verb|u16|. Conversion functions
are provided. In C and C++, many automatic conversions are applied; in
Java and C\#, only widening conversions (that conserve the mathematical
value) are implicit. In T1, all conversions must be explicit: this is
meant to force the developer to have a clear mental picture of what
happens to the data.

\begin{rationale}
For the default integer type, typically used for indexing into arrays,
the following language designs (at least) are possible:
\begin{itemize}

    \item Use a native type of the architecture, with wrap-around
    semantics. This is what Rust does with the \verb|isize| and
    \verb|usize| types. In C, there is an unsigned integer type which
    is what \verb|sizeof| returns and \verb|memcpy()| expects; the
    standard headers give it the name ``\verb|size_t|''.

    \item Use a type with a defined, fixed width, typically 32 bits.
    This is the Java road, with \verb|int|. This implies some
    limitations when machine memory sizes have grown so much that
    indexes beyond $2^{31}$ are no longer ridiculous. In C\#, some
    extensive contorsions were made to allow indexing with both
    \verb|int| and \verb|long|.

    \item Transparently expand integers into \emph{big integers} with
    unlimited range, bounded only by the RAM required to represent such
    values. Python uses this strategy; it is also encountered in many
    Scheme implementations. Computations on small integer values can
    be done without dynamic allocation, but large integer values incur
    some performance loss.

    \item Use a type with a defined range (that may depend on the
    architecture) but detect overflows and transform them into actual
    errors. This is reminiscent of Ada.

    \item Make something weird like JavaScript: there are no integers,
    only floating point values. When going out of range, values become
    approximate.

\end{itemize}

T1 follows the Ada way for several reasons:
\begin{itemize}

    \item Wrap-around semantics, or the C ``undefined behaviour'' when
    exceeding range on a signed type, make for devious bugs which are
    often security issues. Secure code must often include extensive
    analysis and explicit checks to make sure that overflows don't occur;
    it is safer to make all checks by default, and possibly suppress
    them when the compiler can be convinced that no overflow occurs. We
    may say that integer overflow checks are needed in the same way as
    bounds checking is needed for array accesses.

    \item A fixed-width type is too limiting; e.g. a 32-bit type is
    too expensive for small 16-bit microcontrollers, and yet not large
    enough for large 64-bit systems.

    \item Big integers are seductive but incur some extra costs; in
    particular, some form of dynamic memory allocation is needed, and
    this goes contrary to the strict RAM discipline and memory safety
    that T1 strives to achieve. Moreover, big integers cannot be
    implemented in constant-time.

\end{itemize}

T1 implementations may use a slightly smaller range than expected. For
instance, on a 32-bit architecture, \verb|std::int| values may have a
range limited to $-2^{30}$ to $2^{30}-1$, i.e. 31 bits with signed
interpretation, not 32 bits. This is done in order to allow an in-memory
representation that is compatible with pointers: pointers to instances
are normally aligned, hence use even 32-bit values; thus, integers can
be represented as odd values, the least significant bit being used to
mark the value as an integer. That kind of trick is common in Scheme and
OCaml implementations; it allows storing integers in pointer fields in a
way which works well with runtime type checks and garbage collection.
\end{rationale}

\subsection{Strings}

There is no dedicated ``character string'' type. Strings are arrays of
bytes. Literal strings define statically allocated arrays of bytes. The
name ``\verb|string|'' is provided as an alias for the name of the type
for an array of bytes (which is ``\verb|(std::u8 std::array)|'', as
will be described later on).

\begin{rationale}
In C, there are no real character strings, only arrays of \verb|char|
with a terminating zero. T1 does not need or use a terminating zero,
because its arrays have a definite length accessible at runtime.

In the infancy of computers, ``characters'' were believed to be simple,
atomic elements that could be represented with a simple, small,
fixed-width type, e.g. ``\verb|char|'' in the C language. It soon
appeared that different languages required more characters, and ``code
pages'' were invented to incarnate the interpretation of bytes into
characters. Code pages implied huge interoperability issues, and the
problems were made much worse when non-alphabetic scripts such as
Japanese had to be taken into account.

Unicode is a unifying effort that tries to remove the need for code
pages. Unicode defines \emph{code points}; initially, each code point
was a 16-bit integer, but this proved too limited, and code points can
now use up to 22 bits (valid code point values are in the 0 to 0x10FFFD
range). Java defined its \verb|String| type to be a sequence of
\verb|char| values, a 16-bit type, as per the first version of Unicode.
A more modern redefinition would use a 32-bit integer type, so as to
represent the whole range of possible code points.

This is, however, an illusion. Unicode defines code points, not
``characters''. Consider for instance the English word ``café'' (an
import from French, but still a valid English word); the last character
(``é'') admits two representations in Unicode. The first one is a single
code point \verb|U+00E9| (``latin small letter E with acute''); the
other one is a sequence of two code points, \verb|U+0061| (``latin small
letter E'') followed by \verb|U+0301| (``combining acute accent'').
Thus, a single ``character'' may consist of several code points. A lot
of combinations are possible (in particular in Hangul, the Korean
writing system) and it would not be practical to map all of them into a
single numerical type. Therefore, any sufficiently advanced
Unicode-aware processing of text must be able to accomodate
variable-length representations of characters.

A simpler model is represented by Go: strings are just bytes. When the
bytes must be parsed as text, then they are decoded as per UTF-8 rules.
UTF-8 has some nice properties:
\begin{itemize}

    \item Every ASCII character is encoded as a single byte whose value
    matches the ASCII code of the character. For instance, the ASCII
    code of ``e'' is 0x61, and it is encoded as a single byte of value
    0x61. Thus, ASCII is ``preserved'' by UTF-8 encoding.

    \item All other code points are encoded as sequences of bytes
    with values no less than 0x80; none of these bytes may be
    misinterpreted as an ASCII character.

\end{itemize}
Using UTF-8 means that technical processing, in particular for
text-based protocols such as HTTP, can use the traditional
one-character-per-byte model, provided that the processing uses only
ASCII characters, and bytes with values 0x80 or more are just kept
together. This has the additional benefit of not \emph{enforcing} UTF-8
encoding; this is handy when, for instance, exploring file directories,
where file names are OS-provided sequences of bytes which need not be
valid UTF-8.

T1 follows the Go way and uses byte arrays for strings.
\end{rationale}

Functions that use and return strings assume immutability: a string
instance should never change once initialized. However, T1 does not
enforce this property: when an array of bytes is used as a string, it is
up to the developer to refrain from modifying its contents as long as it
is used elsewhere with immutability semantics.

The T1 \emph{compiler} enforces immutability of all statically allocated
instances, and this includes the instances corresponding to literal
strings.

\begin{rationale}
Enforced immutability would make programming ``safer'' at the expense
of extra allocations. For instance, if bytes are read from a network
interface, then these bytes are written into a buffer. To interpret that
buffer as an immutable string, there are several options:
\begin{itemize}

    \item Copy the bytes into a newly allocated read-only object. This
    is what is done in Go when a ``\verb|[]byte|'' value is cast into
    a ``\verb|string|''. Such a mechanism requires dynamic allocation.

    \item ``Lock'' the buffer with a flag checked at runtime for each
    write access. This requires room for that extra flag, and, indeed,
    runtime checks, which may have a non-negligible cost. Unlocking
    would have to be performed as well. Also, any error will be reported
    only at runtime, which is undesirable in general (compile-time error
    reporting is much preferred).

    \item Use complex borrowing semantics to ensure that concurrent
    modifications don't occur. This is what Rust does, with far-reaching
    consequences on the application structure (what is deemed by the
    colloquial euphemism ``fighting the borrower'').

    \item Don't do anything; just \emph{document} that a given string,
    when provided to a function, may be retained and used after that
    function has returned, and therefore must not be modified.

\end{itemize}
T1 follows the last of these options, based on my own development
experience: I don't tend to make bugs related to immutability confusion,
and thus the enforced extra safety does not seem to be worth the extra
costs for that property. This is a personal judgement call, and I
might add a truly immutable string type in a later version.
\end{rationale}

\subsection{Structures}

New types are built as \emph{structures}. A structure contains
\emph{fields} and \emph{embedded sub-structures}. A field contains a
value (i.e. a reference); an embedded sub-structure is an instance of
another type, which is created along with the encapsulating instance.
Though implementations may vary, the intended effect is that fields
appear in the memory layout in the order they appear in the structure,
and embedded sub-structures are really embedded, i.e. use for their own
memory layout the corresponding chunk of memory of the encapsulating
structure.

Each field has a type, which is a filter on possible values of the
field: these values must have types which are sub-types of the field
type. This is a side-effect; the primary function of the field type is
to qualify the declaration of accessor functions for that type. Embedded
structures also have a type, which defines which structure is embedded.
Only other structures may be embedded; it is not possible to embed basic
types (and it would not make much sense either). Embedding is acyclic: a
structure may not directly or indirectly embed itself.

Arrays of fields and arrays of embedded structures can be defined, with
a fixed number of elements.

Consider the following example. We suppose that the current namespace
is ``\verb|def|'', and that the ``\verb|bar|'' type is defined, or
to-be-defined, in that namespace.
\begin{verbatim}
    struct foo
        x int
        b1 bar
        b2 && bar
        c1 16 bar
        c2 && 16 bar
    end
\end{verbatim}

A structure named \verb|def::foo| is defined, with the following contents:
\begin{itemize}

    \item A field called \verb|def::x|, that may contain values of type
    \verb|std::int| or sub-types thereof (but there cannot be strict
    sub-types of \verb|std::int|). Note that the unqualified name
    ``\verb|int|'' is converted by an active alias to ``\verb|std::int|'',
    because that alias is part of the export list from the namespace
    \verb|std|, which is imported by default.

    \item A field called \verb|def::b1| for references to instances of
    \verb|def::bar| (or sub-types thereof).

    \item An embedded structure of type \verb|def::bar|, with name
    \verb|def::b2|.

    \item An embedded array of 16 references of type \verb|def::bar|,
    with name \verb|def::c1|.

    \item An embedded array of 16 embedded sub-structures of type 
    \verb|def::bar|, with name \verb|def::c2|.

\end{itemize}

No two elements of a structure may have the same name, regardless of
their respective kinds.

\subsubsection{Closing}

When first encountered, a structure type is created in an ``open''
state. This means that its name becomes known, but the full contents are
not yet defined. As long as a structure is open, new fields and embedded
elements (sub-structures, arrays, and arrays of embedded sub-structures)
can be added to the structure type. Once the type is \emph{closed}, no
new contents may be added. In the example above, \verb|def::foo| is
still open: new fields and embedded elements could still be added to the
structure. The ``\verb|end|'' keyword does not close the structure; it
merely exits the syntactic construction that is used to add elements to
a structure.

Similarly, the \verb|def::bar| structure, if not yet defined at this
point of the source code, has been automatically created, in open state
and with no initial contents, when the name ``\verb|def::bar|'' was
first encountered (i.e. when the field \verb|def::b1| was
defined).\footnote{Strictly speaking, the \texttt{\textbf{def::bar}}
structure, if created at this point, is in \emph{implicit} open state;
if it was not explicitly defined at the time \texttt{\textbf{def::foo}}
is closed, then an error is reported, under the assumption that a type
which was never explicitly defined anywhere is probably a typing
mistake.}

A structure will be closed in the following circumstances:
\begin{itemize}

    \item When closed explicitly with a specific function call on the
    type instance (i.e. the instance of type ``\verb|std::type|'' that
    represents this type).

    \item When an instance of the structure is created. Instance creation
    implies memory allocation, which needs the layout to be fixed.

    \item When an encapsulating structure is closed. For a structure to
    be closed, all the structures it embeds must first be closed. Since
    the embedding relationship is acyclic, this process converges.

    \item When the structure is made part of code being compiled.

\end{itemize}

Conversely, sub-types and super-types can be added to a closed
structure; that is, even after \verb|def::foo| is closed, new structures
can be defined and made sub-types of \verb|def::foo|, and
\verb|def::foo| itself can be made sub-types of other structures; these
additions are immediately inherited by existing instances of
\verb|def::foo|.

\subsubsection{Instantiation}

When a structure type \verb|T| is defined, a function with the same name
is created. That function takes no parameter, and returns an instance of
\verb|std::type| which represents the type \verb|T|.

A dedicated function \verb|std::new| takes as parameter a
\verb|std::type| instance, and creates a new instance of the represented
type. The fields of the new instance are set to uninitialized state
(except fields of boolean or modular integer types, which are set to
their default \verb|false| or zero values); this also applies,
recursively, to all embedded elements.

Calling \verb|std::new| on \verb|std::type| instances that do not
represent structure types triggers an exception.

\subsubsection{Accessors}

Structure contents can only be inspected and altered through dedicated
\emph{accessor functions}. These special functions are created when the
structure is closed. The accessors use the element names, depending on
the kind of element:
\begin{itemize}

    \item For a field of name \verb|def::x|, the functions \verb|def::x|
    and \verb|def::->x| are defined, to read and write values to the
    field \verb|def::x| of an instance, respectively. The accessor
    \verb|def::Z->x| clears the field, i.e. sets it to uninitialized
    state. The accessor \verb|def::x?| tests whether the field is
    initialized or not.

    \item For an embedded structure of name \verb|def::x|, one accessor
    function of name \verb|def::x&| is defined, which takes as input
    a reference to an instance of the encapsulating structure, and returns
    a reference to the instance embedded within it.

    \item For an array of references, with name \verb|def::x|, the
    functions \verb|def::x@| and \verb|def::->x@| read and write values
    into the array slot indexed by a given \verb|std::int| value.
    Also, \verb|def::Z->x@| clears a slot, and \verb|def::x@?| tests
    its initialization status. Finally, \verb|def::x*| initializes an
    array instance (of the right type) to provide an array view of the
    references.

    \item For an array of embedded sub-structures, with name \verb|def::x|,
    the \verb|def::x@&| accessor returns a reference to one of the
    embedded sub-structures, by \verb|std::int| index, and \verb|def::x*|
    initializes an array view of the sub-structures.

\end{itemize}

Fields, and slots in embedded arrays of references, are initially
uninitialized. There is no null value; reading an uninitialized field
triggers a runtime error.

For booleans and modular integers, i.e. the \verb|std::iXX| and
\verb|std::uXX| types, corresponding fields are always initialized.
Their starting value is \verb|false| (for booleans) or zero (for modular
integers), and the clearing accessors restore that value. The test
accessors (\verb|def::x?|, \verb|def::x@?|) then always return
\verb|true|. Note that plain \verb|std::int| fields are not in this
situation, and can be truly uninitialized.

\subsubsection{Extension}

A structure may \emph{extend} another structure; this is a combination
of sub-typing and embedding. When structure \verb|B| extends the
structure \verb|A|:
\begin{itemize}

    \item \verb|B| is defined to be a sub-type of \verb|A|.

    \item \verb|B| embeds an instance of \verb|A|, under the name of
    \verb|A| (that is, the accessor \verb|A&| is defined, that takes
    as input parameter a reference to an instance of \verb|B|, and
    returns a reference to the embedded instance of \verb|A|).

    \item Accessors for elements of \verb|A| can be used on an
    instance of \verb|B|, and will access the corresponding elements
    in the instance of \verb|A| which is embedded in \verb|B|.

\end{itemize}

Since extension is both embedding and sub-typing, it combines the
requirements of both; notably, extension cannot be done in a closed
structure, and the extension relationship is acyclic.

A given structure \verb|B| may directly extend a given structure
\verb|A| only once (this is implied by the fact that the extended
structure is embedded under its own name, and names are unique within
structure contents). However, a structure may extend several other
structures. This is a multiple inheritance model, which is powerful but
implies some ambiguous situations. Suppose, for instance, that:
\begin{itemize}

    \item Structure \verb|A| has a field named \verb|x|.

    \item Structure \verb|B| extends \verb|A|.

    \item Structure \verb|C| extends \verb|A|.

    \item Structure \verb|D| extends \verb|B| and \verb|C|.

\end{itemize}
In that situation, the \verb|x| function, which is the read accessor for
the field of the same name in \verb|A|, could be invoked on an instance
of \verb|B| and on an instance of \verb|C|. Since \verb|D| extends
\verb|B|, the accessors that accept an instance of \verb|B| will also
work on an instance of \verb|D|. However, the same can be said about the
accessors that accept an instance of \verb|C|. In fact, since \verb|D|
embeds both a \verb|B| and a \verb|C|, and each embeds an \verb|A|, the
structure \verb|D| indirectly embeds \emph{two} instances of \verb|A|,
and it is unclear which one is supposed to be used when reading the
field \verb|x|. Therefore, invoking \verb|x| on an instance of \verb|D|
triggers an error.\footnote{As we shall see, part of the work of the
compiler is to prove that such an error cannot happen in a given piece
of code.}

It is possible to make an explicit decision, by defining a function
called \verb|x|, attached to type \verb|D|, which then selects the
instance to use:
\begin{verbatim}
    : x (D) B& x ;
\end{verbatim}
This snippet reads as follows:
\begin{itemize}

    \item The ``\verb|:|'' token starts the definition of a new function.
    It is followed by the name of that function (\verb|x|).

    \item The ``\verb|(D)|'' expression registers the new function to
    the type ``\verb|D|''; that is, if a function call for name \verb|x|
    is encountered, and at that time the top element on the stack has
    type \verb|D|, then this function shall be called (and not, in
    particular, the accessor function which is registered on type
    \verb|A|).

    \item Afterwards follows the function body, which here consists in two
    successive function calls: \verb|B&|, which returns a reference to the
    sub-structure embedded in \verb|D| under the name \verb|B|, and then
    \verb|x|. Since that \verb|x| call will operate on the \verb|B|
    instance returned by \verb|B&|, it will use the \verb|A| instance
    embedded in that \verb|B| instance, and not the one embedded in the
    \verb|C| instance which is also embedded in \verb|D|.

    \item The semicolon token (``\verb|;|'') terminates the function body.

\end{itemize}
Thus, this new function explicitly chooses \verb|B|, not \verb|C|. Since
it is registered with type \verb|D|, which is a sub-type of \verb|A|, it
will have precedence over the \verb|x| accessor function defined on
\verb|A|, when invoked over an instance of \verb|D|.

\begin{rationale}
Type extension in T1 maps to the Java extension of classes, while
sub-typing corresponds to the Java extension of interfaces. Historically,
Java had only classes, and interfaces were added afterwards to compensate
for the lack of multiple inheritance. The Java inheritance has several
facets:
\begin{itemize}

    \item Inheritance of storage: state held in the superclass is also
    contained in the subclass instance. In T1, this is done with extension.

    \item Inheritance of behaviour: methods attached on the superclass
    also work with subclasses. This is what sub-typing provides in T1.

\end{itemize}
\end{rationale}

\subsection{Arrays}

Array types are defined on-demand. For a given type \verb|T|, the type
``array of \verb|T|'' has the name:
\begin{verbatim}
    (T std::array)
\end{verbatim}
(including the parentheses). This name is not a name token, as returned
by the lexer; however, as will be explained in the description of the
interpreter, the array type name mimics a sequence of code that, when
processed by the interpreter, yields a reference to the \verb|std::type|
instance that represents the array type. In fact, the \verb|std::array|
function \emph{creates} the array type if it does not already exist, and
registers all accessor functions for array instances.

Array instances really are \emph{views} on a chunk of memory. An
instance of the array type, when instantiated but not initialized,
points to nothing, and calling data accessors triggers an exception.
An array instance is populated in basically four ways:
\begin{itemize}

    \item Make the array instance point to a sequence of values or
    embedded structures within a given structure instance. If the
    sequence of values or embedded structures was declared with the name
    \verb|def::x|, then this is done with the \verb|def::x*| accessor
    function.

    \item Make the array instance point to a locally allocated sequence
    of values or embedded structures. For a local name \verb|x|, this is
    done by using the \verb|x*| pseudo-function name.

    \item Dynamically allocate a new memory chunk, with a specified
    length. When dynamic memory allocation is supported, this is done
    with the \verb|std::make| function.

    \item Initialize the array instance as a sub-array of another array
    instance. The sub-array must be entirely contained within the source
    array. This is done with the \verb|std::sub| function. An array
    instance can be reinitialized as a sub-array of itself with
    \verb|std::subself| (which is just a shorthand for \verb|std::sub|
    with the instance used for both operands).

\end{itemize}

There is thus always an indirection layer when accessing memory chunks.
Memory chunks themselves are not objects, i.e. they cannot be accessed
directly, and do not have a T1 type. Native code called from T1 can
obtain a direct pointer to the data, subject to some caveats (in
particular, objects may be moved in memory by the garbage collector, if
used; and locally allocated objects cease to exist when the owner
function returns): T1 memory safety guarantees that all array accesses
are ``safe'' (e.g. out-of-bounds accesses trigger a runtime error, and
all reachable objects are maintained in memory to avoid dangling
pointers), but native code can bypass such safety features.

\begin{rationale}
The Go language has \emph{arrays} and \emph{slices}. An array is a
sequence of value, and a slice is a view on such a sequence. Most
operations that work on arrays also work on slices. In T1, the Go arrays
become ``chunks of memory'' and are not directly accessible; the T1
``arrays'' are equivalent to the Go slices.

Thus, a T1 array type can be thought of as a structure with three
fields: pointer to the actual object that contains the data, offset and
length of the chunk within that hidden object. There is no notion of
``capacity'' as in Go (T1 arrays are not intrinsically growable).
\end{rationale}

An ``array of \verb|T|'' is an array of references (to elements of type
\verb|T|, or sub-types thereof). A newly created memory chunk will have
all slots uninitialized (or set to \verb|false| or zero, for booleans
and modular integers). The following accessor functions are defined:
\begin{itemize}

    \item \verb|std::make| dynamically allocates a new memory chunk,
    and initializes the array instance to point to that chunk.

    \item \verb|std::sub| initializes an array as a view of a chunk of
    another array. The source array must have been initialized.

    \item \verb|std::subself| merely duplicates the argument, then calls
    \verb|std::sub|.

    \item \verb|std::init?| returns \verb|true| if the array instance
    was initialized, \verb|false| otherwise. If the array instance was
    not initialized, then calls to the other functions below will
    trigger a runtime error.

    \item \verb|std::length| returns the length of the array (number of
    elements).

    \item \verb|std::@| and \verb|std::->@| read and write a value from
    an array slot or to an array slot, respectively, indexed by an
    \verb|std::int| value. Array indexes start at zero.

    \item \verb|std::Z->@| clears an array slot, and \verb|std::@?|
    returns its initialization status. For an array of booleans or
    modular integers, clearing a slot means setting it to \verb|false|
    or zero, and \verb|std::@?| always returns \verb|true|.

\end{itemize}

Types for arrays of embedded structures can also be obtained with
\verb|std::array&|. The expression:
\begin{verbatim}
    (T std::array&)
\end{verbatim}
will return an array type that, when instantiated and initialized, is
a view to a sequence of contiguously allocated instances of \verb|T|.
For such an array, the accessors that use or return references
(\verb|std::@|, \verb|std::->@|, \verb|std::Z->@| and \verb|std::@?|)
are not defined; instead, the accessor \verb|std::@&| returns a
reference to one of the structures embedded in the array.

Any other type can present an array-like interface by defining the
appropriate methods. It shall be noted that support for the
\verb|std::sub| function implies that any array-like type must
be able to present some of its contents in a contiguous sequence in
memory.

\begin{rationale}
The main idea behind arrays-as-views is to make it so that any function
that can work on arrays will also work on a sub-array. In practical Java
or C\# code that processes binary data (e.g. I/O code), several methods
are often needed, e.g. a \verb|write()| that takes as parameter an array
of bytes, and another \verb|write()| method that takes as parameters an
array of bytes, a start offset and a length. Array views are meant to
avoid these multiple methods. Moreover, they allow user-defined types
with array-like interfaces to be used instead (e.g. a growable vector of
bytes).
\end{rationale}

\subsection{Generics}

The on-demand creation of array types is an example of how generic types
are managed in T1. In full generality, container types are meant to be
created with metaprogramming. For instance, growable array types are
created with \verb|std::list|. The following sequence of code:
\begin{verbatim}
    (u8 list)
\end{verbatim}
will return an \verb|std::type| instance that represents growable arrays
of bytes. This is a normal structure type (albeit with a name that is
not a qualified name token), and the \verb|new| function on that type
instance will return a new growable array of bytes with an initial size
of zero. Each growable array type is a sub-type of the corresponding
array type, and offers the relevant accessor functions, as well as some
extra functions to append or remove elements.

Syntactic facilities are made available to users, in order to define
their own generic types. This is not restricted to making new types
parametrized by other types; this is more an expression that source code
can define functions and invoke them during the interpretation itself,
to process further source code and define other functions in arbitrary
ways.

\begin{rationale}
In Java, an important point of generics is that they impact the type
analysis, but do not create new types: \verb|ArrayList<String>| and
\verb|ArrayList<Date>| both use the same \verb|Class| instance (their
respective \verb|getClass()| methods return the same object), and cannot
be distinguished from each other at runtime. This is an historical
consequence of generics being added only relatively late in the language
(for Java 5). The generics are handled as an extra layer at compile-time
that is used to avoid having the developer make explicit casts and risk
the dreaded \verb|ClassCastException|.

Conversely, in C\#, \verb|List<string>| and \verb|List<DateTime>| are
distinct types with distinct runtime \verb|Type| instances (as returned
by \verb|GetType()|). The C\# compiler analyzes the source code to make
sure that, when replacing the type parameters with actual types (that
comply with the expressed restrictions on the type parameters), the
resulting code will still be valid; but the runtime machine will create
as many distinct types as necessary. T1 works similary to C\#, minus the
initial abstract analysis: in T1, we do not really care whether things
\emph{would} work with some large categories of types, but whether they
\emph{will} work with the types that the source code actually uses.
\end{rationale}
