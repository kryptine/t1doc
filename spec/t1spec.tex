\documentclass[12pt]{article}
  
\usepackage[lf]{ebgaramond}
\usepackage[cmintegrals,cmbraces]{newtxmath}
\usepackage{ebgaramond-maths}
\usepackage{biolinum}
\usepackage[scaled=0.85]{inconsolata}
%\usepackage[scaled=0.75]{DejaVuSansMono}
\makeatletter
\def\verbatim@font{\ttfamily\bfseries}
\makeatother

\usepackage[T1]{fontenc}
\usepackage[utf8x]{inputenc}
\usepackage[english]{babel}
\usepackage{graphicx}
\usepackage{xspace}

\usepackage{wallpaper}
\usepackage{xcolor}
\usepackage{fancyvrb}
\usepackage{relsize}
\usepackage{multicol}

\usepackage{fullpage}

\usepackage{sectsty}
\allsectionsfont{\sffamily}

\usepackage{hyperref}
\usepackage[all]{hypcap}

%\usepackage{framed}
\usepackage[framemethod=tikz]{mdframed}

\setlength{\parindent}{0pt}
\setlength{\parskip}{2ex}

\definecolor{rationalebg}{rgb}{1.0,1.0,0.8}

%\newenvironment{rationale}{%
%    \begingroup\small%
%    \colorlet{shadecolor}{rationalebg}%
%    \begin{shaded}%
%    \begin{list}{}{%
%        \setlength{\topsep}{0pt}%
%        \setlength{\leftmargin}{3em}%
%        \setlength{\rightmargin}{0pt}%
%        \setlength{\listparindent}{\parindent}%
%        \setlength{\itemindent}{\parindent}%
%        \setlength{\parsep}{\parskip}%
%    }%
%    \item[]}{\end{list}\end{shaded}\endgroup}
\newenvironment{rationale}{%
    \begingroup\small%
    \begin{mdframed}[hidealllines=true,backgroundcolor=rationalebg,leftmargin=3em]}{\end{mdframed}\endgroup}

\title{\textsf{\textbf{The T1 Programming Language}}}
\author{Thomas Pornin}

\begin{document}

\maketitle

\begin{abstract}
\noindent This document describes the design of the T1 programming
language. It is intended to serve both as a specification of the core
language features, and a rationale for the design decisions.

\vspace{2ex}
\noindent T1 is an imperative language that offers memory safety,
generic metaprogramming, strong static typing through whole-program
analysis, and support for object-oriented programming. It is geared
toward efficient support of memory-constrained architecture, and in
particular can be embedded as efficient co-routines with tight bounds
on stack usage. Nevertheless, T1 aims at also being usable as a
general purpose language.

\vspace{2ex}
\noindent \textbf{\textsf{WARNING:}} T1 is a work in progress, and the
specification may be altered as implementation of the compiler unveils
unforeseen difficulties or potential desirable feature changes.
\end{abstract}

\newpage
\section{Overview Of T1}

\subsection{Project Goals}

T1 aims at providing a number of features that are not all obtained
together from existing languages. It can be viewed as an extension of
T0, the Forth-like language which is used for some parts of BearSSL
(namely, for processing handshake messages, and decoding X.509
certificates). Within the context of BearSSL, T0 exists so as to express
complex nested decoding and encoding over streamed data: in order to
explore an encoded object with nested structures without requiring
buffering of the entire object, the decoding process must be
interruptible and restartable, i.e. live as a \emph{coroutine} that can
be scheduled to run when and only when data bytes become available. The
standard C language lacks coroutines; coroutines can be defined in
non-standard ways, but will require an in-memory stack of non-negligible
size for constrained systems; T0 provides a lightweight coroutine that
uses a very small custom stack (with guaranteed limits on maximum stack
growth).

T1 is an evolution of T0 into a more ambitious project:
\begin{itemize}

    \item Like T0, T1 must support compilation into components that can
    be embedded within applications that run on ``bare metal'' systems
    (no OS support).

    \item Lightweight coroutines must be supported.

    \item T1 must be (by default) memory-safe. T0 has only limited
    memory-safety, in that the compiler proves strict bounds on maximum
    stack usage; however, array accesses in T0 are not checked with
    regards to array boundaries.

    \item T1 shall support object-oriented programming. OOP is primarily
    a way to structure application code; in plain C, this is usually
    done with function pointers arranged in vtables. The language can
    provide primitives that help with OOP, for a simpler and safer
    syntax. T0 has no specific OOP features.

    \item T1 should \emph{optionally} support dynamic memory allocation.
    This more or less implies the use of a garbage collector, to
    maintain memory safety. It must be possible to write code that does
    not use the GC, and, when the GC is present, it must be configurable
    with strict limits on allocated size.

    \item Code written in T1 is meant to be embeddable in applications
    that primarily use C; thus, call to T1 code from C, and to C from
    T1, shall be simple and efficient. In particular, the memory layout
    of T1 objects should have a predictable counterpart in the C world,
    so that direct access is feasible and easy.

    \item While T0 has only a single flat namespace, T1 should offer
    some ways to segment the code space into units that limit the risk
    of name collision, especially if several different developers are
    involved.

\end{itemize}

Memory safety and OOP both require the definition of a rich type system,
which may then also be used by the developer to express constraints on
the structure of the application, that will be verified and enforced at
compilation and/or runtime.

Since T1 aims at being a generic purpose language, it should be possible
to write complex applications entirely in T1, starting with the T1
interpreter/compiler itself (as is traditional for language
development).

Programming language design includes the syntax, which sits between a
rational, deterministic machine (the computer) and a definitely less
rational and deterministic human (the developer). As such, the language
necessarily has an aesthetic facet, which cannot be rationally argued
for or against. As a working principle on these matters, I define myself
as the sole judge; T1 should aesthetically please me.

\subsection{Main Features}

T1 combines many features inspired from other programming languages, but
not hitherto found together in a single language. Inspiration has been
drawn from, in no particular order, C, Java, C\#, Forth, Caml, Rust, Go,
and others. Some of the features will be explained in terms of
comparisons with these other languages. In this section, we give an
overview of the main features.

\paragraph{Imperative.} T1 is an imperative language. More generally, it
strives to give to the programmer a clear mental picture of what happens
in the generated code; notably, order of execution of all operations is
duly specified, and should match whenever possible left-to-right reading
order in the source code. Automatic optimizations should be kept at a
minimum (e.g. no automatic vectorization with SIMD instructions). This
can be thought of as a ``no magic policy''.

\paragraph{Fully Specified.} There is no ``undefined behaviour''. All
operations occur in fully specified ways (this is similar to Java, and
unlike C). There may be some platform-dependent characteristics, such as
the possible range of integer values that can serve as array indices
(this corresponds to types such as \verb|size_t| in C, or \verb|usize|
in Rust).

\paragraph{Combined Interpretation / Compilation Model.} The source
code, when processed by the T1 engine, is really executed, as a script.
The \emph{compiler} is a final optional phase that can extract some of
the defined functions and serialize them into an executable form. While
interpretation and post-compilation execution work on the same
functions, they use quite different models:
\begin{itemize}

    \item During interpretation, everything is dynamic. Functions and
    types can be referenced before they are defined; a violation is
    reported only when trying to actually call a function which is not
    yet defined. Interpreted code has full access to API that allow
    creating new types and functions.

    \item Compilation performs a thorough static type analysis that will
    refuse to produce the executable output for code that does not
    comply to strict rules. The point of the rules is to ensure that
    execution won't fail with a type-related error; they can also
    provide guarantees on good memory-wise behaviour. In particular,
    compiled code is not allowed to be recursive, so that maximum stack
    usage can be \emph{a priori} bounded.

\end{itemize}

Processing source code as a script to be executed means that compilation
necessarily involves \emph{executing} the source code. The interpreter
will provide a specific ``sandboxing'' mode which will prevent access to
system-level features such as the network, or files outside of the
source code collection itself. The ability to sandbox potentially
hostile code is not considered a primary feature, but this should
ultimately be supported.

\paragraph{Extensible Postfix Syntax.} T1 uses a Forth-like syntax, which
relies on postfix notation: operations appear after the operands. While
this syntax does not follow traditional mathematical practice, and is
thus harder to read and understand (at least without training), it has
other advantages that are important to the T1 model:
\begin{itemize}

    \item In the postfix notation, everything happens in left-to-right
    source order. This participates to the no-magic policy.

    \item Since functions work over a shared data stack, they naturally
    receive several arguments and return several values without any
    extra syntax to that effect.

    \item As in Forth, ``immediate'' functions can be defined, that are
    invoked when encountered, in the middle of source code translation.
    This way, the source code itself can take over the interpretation
    process at any time and access the remaining of the source code in
    arbitrary ways. This allows defining new syntax on the fly, and,
    more generally, opens the way to generic powerful
    \emph{metaprogramming}.

    \item Postfix source code can be readily serialized into an
    executable format running on \emph{threaded code}, a well-known code
    generation method that can allow for a very small compiled code
    footprint.

\end{itemize}

\paragraph{Generic Object-Oriented Programming.} In classic OOP (as in
for instance Java or Go), functions can be attached to an object type
and be invoked on an object instance (we then call them ``methods'').
Several functions may share the same name; the one which will be invoked
will depend on the \emph{runtime} type of the object on which the method
is invoked (in C++ and C\# terms, this is how virtual methods work). T1
goes one step further, in that the invoked function may depend on the
runtime types of \emph{all} arguments, not just the first one. Indeed,
while Java, C\# and other languages syntactically single out the first
argument as ``the instance on which the method is invoked'', T1
considers all arguments on the same ground.

\paragraph{Only Dynamic Types.} Consider the following Java code snippet:
\begin{verbatim}
    class A {
        void foo(A a) {
            System.out.println("foo AA");
        }
        void foo(B b) {
            System.out.println("foo AB");
        }
    }
    class B extends A {
        void foo(A a) {
            System.out.println("foo BA");
        }
        void foo(B b) {
            System.out.println("foo BB");
        }
    }
    class C {
        public static void main(String[] args) {
            A x = new B();
            A y = new B();
            x.foo(y);
        }
    }
\end{verbatim}
This code will print ``\verb+foo BA+''. The variable \verb+x+ contains
a reference to an object of type \verb+B+, so the invoked method will
be one of \verb+B+, not one of \verb+A+, even though \verb+x+ was declared
as a variable of type \verb+A+. On the other hand, \verb+y+ was also
declared as a variable of type \verb+A+, and this is the type which will
be used to decide which of \verb+B+'s \verb+foo()+ methods is invoked,
even if the value which is then passed as parameter is really a reference
to an instance of \verb+B+.

This code snippet illustrates that Java uses two distinct notions of type
for purposes of issuing calls to methods:
\begin{itemize}

    \item For the first parameter, i.e. syntactically the instance ``on
    which'' the method is called, its \emph{dynamic type} is used: this
    is the type of the instance, regardless of the apparent type of the
    expression that yields a reference to that object.

    \item For the other parameters (the ones within the parenthesized
    list), only the \emph{static type} is used: this is the type
    syntactically attached to the expression, irrespective of the actual
    value at call time.

\end{itemize}
In T1, this duality is rejected; only dynamic types are used. This is
part of the goal of OOP genericity: if all parameters to a function are
treated on an equal basis, then they should all use the same kind of
type analysis for purposes of method dispatch.

Use of only dynamic types does not mean that static typing cannot be
performed; in fact, the T1 compiler is all about making thorough static
type analysis. Rather, this means that the goal of static analysis is to
work out the possible dynamic types of the parameters upon execution,
and verify that there will indeed be methods matching each call. The
\emph{semantics} of the language are still defined in terms of the
dynamic types of values, not of static types attached to expressions.

\subsection{Memory Model}

All \emph{values} are references, i.e. pointers to object instances.
This also holds, formally, for small integer types; e.g. a value of type
\verb+u32+ (32-bit unsigned integer) that contains the number \verb+5+
is considered to be a pointer to an immutable instance that incarnates
that number. In practice, for small integer types and booleans, these
immutable instances don't actually exist in memory; formally, the
booleans and small integers are still references.

There is no null pointer. When an object is created in memory, its
fields are \emph{uninitialized}, and reading an uninitialized field
triggers a runtime error.

\begin{rationale}
Tony Hoare introduced null references in ALGOL, mostly because it was
easy to do so. He now calls this decision ``his billion-dollar
mistake''. Null pointers imply the risk of null pointer dereference. In
modern ``big'' systems, in which there is an active memory management
unit, a null pointer dereference will reliably trigger a CPU exception,
which the operating system will convert into some sort of interruption
(e.g. a \verb|SIGSEGV| signal on Unix-like systems). However, there can
still be issues when the null pointer is taken as an array, with a large
access index: the offset may make the access valid again, from the point
of view of the MMU. On smaller systems without a MMU, null pointer
dereferences cannot easily be trapped, leading to hard to debug errors.

Some languages do not have null pointers, in particular the Caml family.
These languages demonstrate that avoidance of null pointers is possible
and not especially hard, though it requires explicit initializers for
all fields. T1 takes a ``middle path'' in which null pointers don't
exist, but individual object fields may be uninitialized; this means
that runtime checks will happen only upon field access, not on all
pointer following actions. Static analysis might also be able to remove
some of these checks.
\end{rationale}

There are no C\#-style ``value types''. In C\#, a value type is a
\verb+struct+ that contains fields, and which is cloned when needed. The
main reason to have value types is storage efficiency. Consider, for
instance, an application that manages a large array of dates. In C\#,
this would use an array of the value type \verb+DateTime+; all these
instances would be concatenated in memory into a single allocated chunk.
In Java, which does not have value types, an array of \verb+Date+ would
be used, but this would really be an array of references to individually
allocated \verb+Date+ instances. This would be likely to induce a much
larger overhead for the memory allocator; it also increases access cost,
since that is one extra layer of pointers to follow.

In order to recapture the storage efficiency of value types, T1 defines
\emph{embedding}: when an object type is described, or an array created,
fields can be defined to be either values (i.e. references), or embedded
sub-objects. An embedded sub-object is allocated within the
encapsulating object, and there is no extra pointer. This changes the
semantics (the embedded object is always there, and cannot be
substituted for another), and thus is made explicit in the language;
this is not an automatic optimization.

Since there are no value types, all parameters to functions are
references, and all returned values are references as well. These values
are exchanged over a common \emph{stack}. This stack is separate from
the in-memory structure that keeps track of function calls (in Forth
terms, the data stack is not the system stack). In compiled code, thanks
to the restrictions imposed by the compiler, the stack needs not be more
than a transient abstraction; there is not necessarily a single
dedicated memory area with a stack pointer.

Apart from the stack, functions may also declare local variables and
locally allocated object instances (i.e. on the ``system stack'' in
Forth terms). Local variables contain values, i.e. references to
instances. Local variables are created when the function is entered, and
destroyed when the function exits; notably, they are not bound to scopes
smaller than a function body.

The Go language supports creating structures that can contain either
embedded sub-objects, or pointers to other objects. The default
(simplest) syntax in Go is for embedding, and Go supports value types in
the C\# sense. Since T1 core values are references, the syntax is
different: structure types are by default references, and an extra
syntax is used to make embeddings.

During interpretation, instances are allocated dynamically, and memory
is managed with a garbage collector: unreachable objects are
automatically reclaimed. Compiled code offers several options:
\begin{itemize}

    \item The compilation phase may involve static allocation of
    instances which were created during interpretation, and are
    referenced from the produced code.

    \item Scope-based allocation is possible (i.e. ``on the stack'').
    This is allowed by the compiler only insofar as it can statically
    determine, through escape analysis, that the object will not ever be
    used after the declaring scope has exited; moreover, for objects
    with a variable length (e.g. arrays), the actual length must be
    fixed at compile time. Such allocation does not necessarily happen
    on a physical stack.

    \item Dynamic allocation with automatic reclamation by the GC is
    possible. A point of T1, though, is to make such allocation optional
    (if the code does not use such allocation, the GC itself won't be
    included in the output).

\end{itemize}

All accesses to instances ultimately use special \emph{accessor
functions} which are automatically defined when the corresponding type
is declared. When accessing array elements (by index), the accessor
functions enforce strict bounds checking.

\newpage
\section{Lexing}\label{lexing}

We describe here how the T1 interpreter breaks down source code into
individual tokens, upon which the T1 syntax is built. Since T1 is
generically extensible (interpreted source code can, at any point,
invoke itself and take over processing of the remaining of the source
code), an arbitrary number of new parsing rules can be implemented by
source code. The rules detailed below explain how parsing is done at the
start of the source code processing.

\begin{rationale}
In Forth, the only lexing process is to aggregate sequences of non-space
characters into ``words''. All other syntax, e.g. literal strings or
comments, is implemented by custom ``immediate words'', which are
functions that are invoked right away when encountered in the source
stream. T1 implements a more complicated lexing process for ease of
development.

It shall be noted that Forth aims at offering support for development
right on the target system, which may be embedded and constrained; this
is one of the main reason for the very simple lexing process of Forth.
In T1, the normal model is to make development on a dedicated powerful
development workstation, distinct from the target system on which the
code will run, which is why more expensive lexing is not an issue.
\end{rationale}

\paragraph{Input Characters.} Source code consists in a number of text
streams that are processed in due order. They are normally stored as
individual files. Each stream contains bytes which are interpreted into
\emph{characters} using UTF-8 encoding; in this specification, a
character is a Unicode ``code point'', i.e. an integer in the
\verb|U+0000| to \verb|U+10FFFD| range. Outside of literal strings, only
ASCII characters (\verb|U+0000| to \verb|U+007E|) may appear.

\begin{rationale}

As will be described below, all the lexing really operates on bytes. In
UTF-8 encoding, each ASCII character is encoded as a single byte with
the same value, and all other code points are encoded as sequences of
bytes of value \verb|0x80| or more. The non-ASCII byte values that
appear in literal strings can thus be simply copied, since, as we shall
see, string values are really arrays of bytes. Moreover, the source
itself can define and then invoke functions that can take over source
code processing in arbitrary ways, and such functions may interpret
source bytes differently. In that sense, it is not strictly necessary
that source file uses UTF-8 encoding, only that the parts that rely on
the lexing process described here use only ASCII characters outside of
literal strings.

However, it is expected that most source code writing will be done with
text editors, that are likely to rely on, and enforce, a specific
encoding charset. In the interest of maximum interoperability, it is
here defined that all source code \emph{shall} be UTF-8 encoded, and
interpreters/compilers may enforce it.
\end{rationale}

Text breaks down into lines; each line is terminated by a newline
character (\verb|U+000A|). If a line ends with a CR+LF sequence
(\verb|U+000D| followed by \verb|U+000A|), then this is considered to be
equivalent to a single newline character.

\paragraph{Whitespace.} \emph{Whitespace} is any sequence of one or more
characters in the \verb|U+0000| to \verb|U+0020| range, i.e. all ASCII
control characters, and the ASCII space. Thus, tabulations
(\verb|U+0009|) and newline (\verb|U+000A|) are whitespace. Whitespace
characters separate tokens, but are otherwise not significant.
Indentation, in particular, is a purely aesthetic choice with no impact
on semantics. Take care that whitespace characters appearing within
literal strings do not count as whitespace.

\paragraph{Comments.} The character ``\verb|#|'' starts a comment
(unless it appears within a literal string or a character constant). The
comment spans to the end of the current line, but does not include the
newline character that terminates that line. If a comment appears on the
last line of a file that does not end with a newline character, the
comment spans to the end of the file. Comments are ignored; since, in
general, a comment is immediately followed by a newline character, that
newline character acts as whitespace.

\paragraph{Single-Character Tokens.} Each of the following characters,
when encountered outside of a literal string or character constant, counts
as a token in its own right:
\begin{verbatim}
    ( ) [ ] { } '
\end{verbatim}

\paragraph{Names and Numerical Constants.} The lexer parses a
\emph{word} as a sequence of non-space printable ASCII characters
(\verb|U+0021| to \verb|U+007E|), excluding the following characters:
\begin{verbatim}
    ( ) [ ] { } ' " #
\end{verbatim}
The lexing process is ``greedy'': the longest sequence of allowed
characters is assembled, and stops at the first disallowed character
(from the list above), whitespace, or end-of-stream, whichever comes
first.

\emph{Numerical Constants} include the following:
\begin{itemize}

    \item \emph{Boolean constants} are the words ``\verb|true|'' and
    ``\verb|false|''.

    \item \emph{Character constants} are all words that start with a
    backquote character (``\verb|`|'', \verb|U+0060|).

    \item \emph{Number constants} are all the words that start with an
    ASCII digit (``\verb|0|'' to ``\verb|9|''), or a minus
    (``\verb|-|'') or plus sign (``\verb|+|'') followed by an ASCII
    digit.

\end{itemize}

\emph{Names} are words which are not numerical constants.

Note that a word which starts with a sequence that introduces a numerical
constant, but fails to parse as a valid numerical constant, triggers an
error; it is not ``demoted'' to being a name.

\begin{rationale}
In Forth, when a word is encountered, it is first evaluated as a
function name; this works because Forth uses a strict define-before-use
policy, so any word can be unambiguously matched against existing
functions at this point. Only words which are not recognized as function
names will be re-interpreted as possible numerical constants. A side
effect is that it allows defining functions with names such as
\verb+42+, a feature which is more confusing than useful.

In T1, we allow referencing functions and types that will be defined
later on, and thus we cannot use numerical interpretation as a fallback
for unknown function names. The non-reinterpretation of invalid
numerical constants as names is meant to promote readability: looking at
the start of a word is enough to know whether it is a numerical constant
or a name; it also allows later versions of T1 to enrich the syntax with
more numerical constants, e.g. floating-point values, without breaking
backward compatibility.

A consequence is that function names cannot start with a digit, such as
Forth's ``\verb+2DUP+''.
\end{rationale}

\paragraph{Number Constants.} Valid number constants are:
\begin{itemize}

    \item a sequence of ASCII digits, interpreted as an integer value
    in base 10;

    \item the sequence ``\verb|0x|'' or ``\verb|0X|'', followed by
    one or more hexadecimal digits (hexadecimal digits are ASCII digits
    ``\verb|0|'' to ``\verb|9|'', ASCII uppercase letters
    ``\verb|A|'' to ``\verb|F|'', and ASCII lowercase letters
    ``\verb|a|'' to ``\verb|f|''), interpreted as an integer value
    in base 16 (case is not significant);

    \item the sequence ``\verb|0b|'' or ``\verb|0B|'', followed by one
    or more binary digits (``\verb|0|'' or ``\verb|1|''), interpreted as
    an integer value in base 2;

    \item any of the above, preceded by a minus sign (``\verb|-|'') or a
    plus sign (``\verb|+|''); the minus sign makes the value negative,
    while the plus sign does not change the value and is purely
    cosmetic;

    \item any of the above, followed by a suffix in the following list,
    and defining the constant to have the corresponding modular integer
    type: \verb|i8| \ \verb|i16| \ \verb|i32| \ \verb|i64| \ \verb|u8|
    \ \verb|u16| \ \verb|u32| \ \verb|u64|

\end{itemize}

If the number constant does not have an explicit type suffix, then it
has plain integer type (\verb|std::int|, as will be defined in
section~\ref{types}). If the value does not fit in the allowed range for
the target type, then an error is raised.

\paragraph{Character Constants.} A \emph{character constant} describes
an integer value of type \verb|std::u8| in the 0 to 126 range
(inclusive). Valid character constants consist in a backquote character
(\verb|U+0060|) followed by:
\begin{itemize}

    \item a single ASCII character in the \verb|U+0021| to
    \verb|U+007E| range, excluding the backslash character
    (``\verb|\|'');

    \item an escape sequence that starts with a backslash, followed
    by one character:
    \begin{itemize}

        \item \verb|\s| stands for space (\verb|U+0020|);
        \item \verb|\t| stands for tabulation (\verb|U+0009|);
        \item \verb|\r| stands for carriage return (\verb|U+000D|);
        \item \verb|\n| stands for newline (\verb|U+000A|);
        \item \verb|\'| stands for quote (\verb|U+0027|);
        \item \verb|\`| stands for backquote (\verb|U+0060|);
        \item \verb|\"| stands for double-quote (\verb|U+0022|);
        \item \verb|\\| stands for backslash (\verb|U+005C|).

    \end{itemize}

\end{itemize}

In all cases, the character constant stands for the numerical value that
corresponds to the represented code point.

\begin{rationale}
Character constants are deliberately limited to plain ASCII because
they have type \verb|std::u8|, following the decision that ``normal''
strings really are sequences of bytes. This will be explained in more
details in section~\ref{types}.
\end{rationale}

Since character constants are self-terminated (inspection of their
contents is enough to decide that no extra character follows in the
token), they need not be separated by whitespace from the next token.
Thus, ``\verb|`ab|'' is parsed as two tokens, the character constant for
lowercase letter ``a'', then the one-character name ``\verb|b|''. This
is of course confusing, so don't do that.

\paragraph{Literal Strings.} A \emph{literal string} represents a value
which is a sequence of bytes. Such a token starts with a double-quote
character ``\verb|"|'' and ends at the next unescaped double-quote
character. The following rules apply:
\begin{itemize}

    \item The starting and ending double-quote characters are not part
    of the string contents.

    \item Bytes appearing in the string literal, other than backslash
    and newline, are part of the string literal. This includes all byte
    values, even ASCII control characters (note that a CR+LF sequence at
    the end of a source text line counts as a single newline character,
    which cannot appear unescaped within a literal string).

    \item When a backslash appears within a literal string:
    \begin{itemize}

        \item If the backslash is immediately followed by the newline
        (or CR+LF) that ends the line, then this is a \emph{line
        escape}: the next line must start with zero or more whitespace
        characters (except newline), followed by a double-quote
        character; the backslash, newline, whitespace and double-quote
        character are then skipped, and parsing of the literal string
        continues after the double-quote character.

        \item Otherwise, the backslash character must begin an escape
        sequence. Escape sequences are:
        \begin{itemize}

            \item escape sequences that may appear in character constants;

            \item \verb|\x| followed by exactly two hexadecimal digits,
            standing for the byte whose value is expressed in hexadecimal
            by these two digits;

            \item \verb|\u| followed by exactly four hexadecimal digits,
            standing for the UTF-8 encoding of the code point whose value
            is expressed in hexadecimal by these four digits;

            \item \verb|\U| followed by exactly six hexadecimal digits,
            standing for the UTF-8 encoding of the code point whose value
            is expressed in hexadecimal by these six digits.

        \end{itemize}

        Note that hexadecimal digits are ASCII digits ``\verb|0|'' to
        ``\verb|9|'', uppercase letters ``\verb|A|'' to ``\verb|F|'',
        and lowercase letters ``\verb|a|'' to ``\verb|f|''. Case is not
        significant for hexadecimal digits. Unrecognized escape
        sequences trigger errors.

    \end{itemize}

\end{itemize}

For instance, this literal string:
\begin{verbatim}
    "Hello\
    " World!"
\end{verbatim}
uses a line escape and has contents ``Hello World!''.

The four following strings:
\begin{verbatim}
    "café"
    "caf\xc3\xa9"
    "caf\u00E9"
    "caf\U0000E9"
\end{verbatim}
all define the same sequence of five bytes. Take care that ``\verb|\x|''
escapes allow inclusion of arbitrary byte values which do not
necessarily correspond to the valid UTF-8 encoding of a sequence of code
points.

Apart from line escapes, newline characters may not appear into a
literal string (but a ``\verb|\n|'' escape sequence can be used to
include a newline character in the string contents). Thus, a literal
string may span several lines only if each line (except the last) ends
with a line escape. Moreover, a string literal must be terminated before
the end of the current text stream; string literals do not span across
files.

Since the whitespace characters that are part of a line escape do not
include the newline character, a comment is not possible within that
whitespace.

\newpage
\section{Names}

In T1, functions, types and local variables have names. A name is a
sequence of characters (unicode code points). In general, any sequence
of code points is usable for any purpose. However, names that are
encountered within source code are subject to some processing which
modifies their interpretation and restricts their syntax.

The lexer parses a name as a token that contains only printable non-space
ASCII characters, excluding a few ``forbidden'' characters (parentheses,
braces...). Such a name may be ``qualified'' or ``unqualified'':
\begin{itemize}

    \item A \emph{qualified name} contains a single instance of the
    sequence ``\verb|::|''. The part before the sequence is the
    \emph{namespace}, and the part after is the \emph{raw
    name}.\footnote{In general a name shall not contain more than one
    instance of ``\texttt{\textbf{::}}''. If it does, then all
    operations that split the name into a namespace and a ``raw'' name
    use the first (leftmost) occurrence of ``\texttt{\textbf{::}}'' as
    splitting point.}

    \item An \emph{unqualified name} does not contain the sequence
    ``\verb|::|''. Thus, raw names (obtained from removing the namespace
    from a qualified name) are unqualified.

\end{itemize}

The general model of source code interpretation is that qualified names
designate a specific entity, and unqualified names are interpreted
depending only on the syntactic context. For instance, under normal
conditions, the interpreter reads the next token and expects it to be
either a numerical constant, a literal string, or a function name. In
this last case:
\begin{itemize}

    \item If the name is qualified, then this designates exactly that
    function.

    \item If the name is not qualified, then it will be matched against
    the following, in due order:
    \begin{itemize}

        \item If the name matches an accessor for a locally allocated
        variable or instance, then the name is interpreted as an
        invocation of that accessor.

        \item If the name is part of one of the currently defined aliases
        (imports), then it designates the function that the alias maps
        to.

        \item Otherwise, the name is considered to use the currently
        active namespace.

    \end{itemize}

\end{itemize}

Namespaces are used to keep track of defined functions and types, and to
avoid spurious collisions. Normal source code should contain very few
qualified names:
\begin{itemize}

    \item At any point in the source code, there is an active namespace
    in which new functions and types are defined.

    \item Access to names in other namespaces is normally done through
    aliases and import lists.

    \item The developers are supposed to have full knowledge of ``their''
    namespaces, and thus avoid internal collisions.

\end{itemize}

There are no visibility rules, i.e. public and non-public functions and
types. Every such element \emph{can} be accessed by using a qualified
name. However, good software engineering practice is to refrain from
doing so in the general case. Functions and types can be added to the
\emph{export list} of the currently active namespace: such an export
list can be imported from other namespaces with an ``\verb|import|''
clause, which locally defines corresponding aliases for the exported
names. In that sense, ``public'' functions can be defined by making them
part of the export list of their namespace, which documents the intent
of making them callable from other namespaces.

More details on import lists and aliases are given in
section~\ref{syntax:namespace}.

All namespaces that start with ``\verb|std|'' are reserved for the T1
implementation. Notably, the export list for ``\verb|std|'' itself is
automatically imported, and it defines aliases for all the core
syntactic constructions and types. Source code can, at any time, clear
the current list of aliases, including those from the ``\verb|std|''
export list.

\newpage
\section{Types}\label{types}

All T1 values are \emph{references} to \emph{instances}. This includes
the basic types (booleans, small integers...). Formally, a value for the
integer ``5'' is considered to be a reference to an immutable object
instance that represents that integer; such instances are virtual and
cannot be really created in memory. This definition allows us to define
the T1 type system without making special cases for such basic types. In
practice, there are some restrictions in compiled code that avoid type
punning that would be expensive to implement.

\subsection{Sub-Typing}

\emph{Sub-typing} is a mechanism which incarnates promises of
functionality. When type \verb|bar| is a sub-type of \verb|foo|, then it
means that whenever a function expects as argument a reference to an
object of type \verb|foo|, it may receive instead a reference to an
object of type \verb|bar|, and things ``should work''. As we shall see
later on, the only thing that can be made with values is to call
functions on them, and function calls are dynamically mapped based on
the runtime types of their arguments; thus, making \verb|bar| a sub-type
of \verb|foo| means that for every function that takes as input a
\verb|foo|, a function of the same name that accepts a \verb|bar| is
defined.

Whether such promises are fulfilled or not does not impact sub-typing.
During interpretation, an unfulfilled promise will trigger a runtime
error at the time the function is called. The compiler statically checks
that such a situation does not occur; in that sense, the compiler does
not check that there are methods for \verb|bar| that correspond to all
methods for \verb|foo|, only that there are such methods for all calls
that can actually occur in the compiled code.

Sub-typing has the following rules:
\begin{itemize}

    \item Every type is considered to be a sub-type of itself. A
    \emph{strict sub-type} of type ``\verb|T|'' is a sub-type of
    ``\verb|T|'' that is not ``\verb|T|'' itself.

    \item All types are sub-types of ``\verb|std::object|''.

    \item Strict sub-typing is an acyclic graph. A given type is
    a sub-type of itself, but shall not be a sub-type of any
    strict sub-type of itself.

    \item Sub-typing rules can be added to any type at any time,
    provided that they don't create cycles. Basic types are an
    exception, in that they cannot be sub-typed, or made strict
    sub-types of any other types except ``\verb|std::object|''. Note
    that sub-typing can be added even on types for which instances have
    already been created.

\end{itemize}

Sub-typing is distinct from both embedding and extensions, which will be
covered later on.

If \verb|A| is a sub-type of \verb|B|, then \verb|B| is a \emph{super-type}
of \verb|A|. A type can have several direct sub-types and several direct
super-types. Sub-types and super-types are not ordered.

\begin{rationale}
Sub-typing is somewhat similar to interfaces in Java and C\#: a
mechanism to tag types, and promise existence of methods, without
actually implementing them. However, in both Java and C\#, such promises
are checked by the compiler: a class that declares that it implements a
given interface, but fails to provide the relevant methods, will be
rejected at compilation. This does not occur in T1, which concentrates
on actual usage: a compilation error is triggered not by failure of
providing a method that matches a given sub-typing relationship, but
by trying to call such a method.
\end{rationale}

\subsection{Basic Types}

\emph{Basic types} are the following:
\begin{itemize}

    \item \verb|std::object| is the root of the sub-typing graph.
    Instances of \verb|std::object| cannot be created. The basic
    equality and inequality functions (``\verb|=|'' and ``\verb|<>|'')
    are defined on \verb|std::object| and implement (in)equality of
    references. Due to the way function lookups are performed, this
    behaviour is inherited by all types, unless explicitly overridden.

    \item \verb|std::bool| is the boolean type. The two possible values
    are ``\verb|true|'' and ``\verb|false|''.

    \item \verb|std::int| is the default integer type. It has a range
    which depends on the current architecture and implementation, but
    which is large enough to serve as index value in arrays. It is
    signed: its range is $-m$ to $m-1$ for a given integer $m = 2^\ell$.
    Operations on \verb|std::int| are checked; any overflow condition
    triggers a runtime error.

    \item \verb|std::u8|, \verb|std::u16|, \verb|std::u32| and
    \verb|std::u64| implement unsigned integers modulo $2^8$, $2^{16}$,
    $2^{32}$ and $2^{64}$, respectively. Since they are modular integers,
    they have ``wrap-around'' semantics, and never overflow.

    \item \verb|std::i8|, \verb|std::i16|, \verb|std::i32| and
    \verb|std::i64| are the signed types corresponding to the unsigned
    modular types. For instance, \verb|std::i8| is for integers in the
    $-128$ to $127$ range. They implement wrap-around semantics, just
    like unsigned types; like Java and unlike C, computations on signed
    types that exceed the range yield well-defined results.

\end{itemize}

\emph{There are no automatic conversions.} Contrary to C, no value can
be used as a boolean, except the values of type \verb|std::bool|.
Arithmetic operations such as addition cannot involve distinct types;
e.g. you cannot add a \verb|u8| to a \verb|u16|. Conversion functions
are provided. In C and C++, many automatic conversions are applied; in
Java and C\#, only widening conversions (that conserve the mathematical
value) are implicit. In T1, all conversions must be explicit: this is
meant to force the developer to have a clear mental picture of what
happens to the data.

\begin{rationale}
For the default integer type, typically used for indexing into arrays,
the following language designs (at least) are possible:
\begin{itemize}

    \item Use a native type of the architecture, with wrap-around
    semantics. This is what Rust does with the \verb|isize| and
    \verb|usize| types. In C, there is an unsigned integer type which
    is what \verb|sizeof| returns and \verb|memcpy()| expects; the
    standard headers give it the name ``\verb|size_t|''.

    \item Use a type with a defined, fixed width, typically 32 bits.
    This is the Java road, with \verb|int|. This implies some
    limitations when machine memory sizes have grown so much that
    indexes beyond $2^{31}$ are no longer ridiculous. In C\#, some
    extensive contorsions were made to allow indexing with both
    \verb|int| and \verb|long|.

    \item Transparently expand integers into \emph{big integers} with
    unlimited range, bounded only by the RAM required to represent such
    values. Python uses this strategy; it is also encountered in many
    Scheme implementations. Computations on small integer values can
    be done without dynamic allocation, but large integer values incur
    some performance loss.

    \item Use a type with a defined range (that may depend on the
    architecture) but detect overflows and transform them into actual
    errors. This is reminiscent of Ada.

    \item Make something weird like JavaScript: there are no integers,
    only floating point values. When going out of range, values become
    approximate.

\end{itemize}

T1 follows the Ada way for several reasons:
\begin{itemize}

    \item Wrap-around semantics, or the C ``undefined behaviour'' when
    exceeding range on a signed type, make for devious bugs which are
    often security issues. Secure code must often include extensive
    analysis and explicit checks to make sure that overflows don't occur;
    it is safer to make all checks by default, and possibly suppress
    them when the compiler can be convinced that no overflow occurs. We
    may say that integer overflow checks are needed in the same way as
    bounds checking is needed for array accesses.

    \item A fixed-width type is too limiting; e.g. a 32-bit type is
    too expensive for small 16-bit microcontrollers, and yet not large
    enough for large 64-bit systems.

    \item Big integers are seductive but incur some extra costs; in
    particular, some form of dynamic memory allocation is needed, and
    this goes contrary to the strict RAM discipline and memory safety
    that T1 strives to achieve. Moreover, big integers cannot be
    implemented in constant-time.

\end{itemize}

T1 implementations may use a slightly smaller range than expected. For
instance, on a 32-bit architecture, \verb|std::int| values may have a
range limited to $-2^{30}$ to $2^{30}-1$, i.e. 31 bits with signed
interpretation, not 32 bits. This is done in order to allow an in-memory
representation that is compatible with pointers: pointers to instances
are normally aligned, hence use even 32-bit values; thus, integers can
be represented as odd values, the least significant bit being used to
mark the value as an integer. That kind of trick is common in Scheme and
OCaml implementations; it allows storing integers in pointer fields in a
way which works well with runtime type checks and garbage collection.
\end{rationale}

\subsection{Strings}

There is no dedicated ``character string'' type. Strings are arrays of
bytes. Literal strings define statically allocated arrays of bytes. The
name ``\verb|string|'' is provided as an alias for the name of the type
for an array of bytes (which is ``\verb|(std::u8 std::array)|'', as
will be described later on).

\begin{rationale}
In C, there are no real character strings, only arrays of \verb|char|
with a terminating zero. T1 does not need or use a terminating zero,
because its arrays have a definite length accessible at runtime.

In the infancy of computers, ``characters'' were believed to be simple,
atomic elements that could be represented with a simple, small,
fixed-width type, e.g. ``\verb|char|'' in the C language. It soon
appeared that different languages required more characters, and ``code
pages'' were invented to incarnate the interpretation of bytes into
characters. Code pages implied huge interoperability issues, and the
problems were made much worse when non-alphabetic scripts such as
Japanese had to be taken into account.

Unicode is a unifying effort that tries to remove the need for code
pages. Unicode defines \emph{code points}; initially, each code point
was a 16-bit integer, but this proved too limited, and code points can
now use up to 22 bits (valid code point values are in the 0 to 0x10FFFD
range). Java defined its \verb|String| type to be a sequence of
\verb|char| values, a 16-bit type, as per the first version of Unicode.
A more modern redefinition would use a 32-bit integer type, so as to
represent the whole range of possible code points.

This is, however, an illusion. Unicode defines code points, not
``characters''. Consider for instance the English word ``café'' (an
import from French, but still a valid English word); the last character
(``é'') admits two representations in Unicode. The first one is a single
code point \verb|U+00E9| (``latin small letter E with acute''); the
other one is a sequence of two code points, \verb|U+0061| (``latin small
letter E'') followed by \verb|U+0301| (``combining acute accent'').
Thus, a single ``character'' may consist of several code points. A lot
of combinations are possible (in particular in Hangul, the Korean
writing system) and it would not be practical to map all of them into a
single numerical type. Therefore, any sufficiently advanced
Unicode-aware processing of text must be able to accomodate
variable-length representations of characters.

A simpler model is represented by Go: strings are just bytes. When the
bytes must be parsed as text, then they are decoded as per UTF-8 rules.
UTF-8 has some nice properties:
\begin{itemize}

    \item Every ASCII character is encoded as a single byte whose value
    matches the ASCII code of the character. For instance, the ASCII
    code of ``e'' is 0x61, and it is encoded as a single byte of value
    0x61. Thus, ASCII is ``preserved'' by UTF-8 encoding.

    \item All other code points are encoded as sequences of bytes
    with values no less than 0x80; none of these bytes may be
    misinterpreted as an ASCII character.

\end{itemize}
Using UTF-8 means that technical processing, in particular for
text-based protocols such as HTTP, can use the traditional
one-character-per-byte model, provided that the processing uses only
ASCII characters, and bytes with values 0x80 or more are just kept
together. This has the additional benefit of not \emph{enforcing} UTF-8
encoding; this is handy when, for instance, exploring file directories,
where file names are OS-provided sequences of bytes which need not be
valid UTF-8.

T1 follows the Go way and uses byte arrays for strings.
\end{rationale}

Functions that use and return strings assume immutability: a string
instance should never change once initialized. However, T1 does not
enforce this property: when an array of bytes is used as a string, it is
up to the developer to refrain from modifying its contents as long as it
is used elsewhere with immutability semantics.

The T1 \emph{compiler} enforces immutability of all statically allocated
instances, and this includes the instances corresponding to literal
strings.

\begin{rationale}
Enforced immutability would make programming ``safer'' at the expense
of extra allocations. For instance, if bytes are read from a network
interface, then these bytes are written into a buffer. To interpret that
buffer as an immutable string, there are several options:
\begin{itemize}

    \item Copy the bytes into a newly allocated read-only object. This
    is what is done in Go when a ``\verb|[]byte|'' value is cast into
    a ``\verb|string|''. Such a mechanism requires dynamic allocation.

    \item ``Lock'' the buffer with a flag checked at runtime for each
    write access. This requires room for that extra flag, and, indeed,
    runtime checks, which may have a non-negligible cost. Unlocking
    would have to be performed as well. Also, any error will be reported
    only at runtime, which is undesirable in general (compile-time error
    reporting is much preferred).

    \item Use complex borrowing semantics to ensure that concurrent
    modifications don't occur. This is what Rust does, with far-reaching
    consequences on the application structure (what is deemed by the
    colloquial euphemism ``fighting the borrower'').

    \item Don't do anything; just \emph{document} that a given string,
    when provided to a function, may be retained and used after that
    function has returned, and therefore must not be modified.

\end{itemize}
T1 follows the last of these options, based on my own development
experience: I don't tend to make bugs related to immutability confusion,
and thus the enforced extra safety does not seem to be worth the extra
costs for that property. This is a personal judgement call, and I
might add a truly immutable string type in a later version.
\end{rationale}

\subsection{Structures}

New types are built as \emph{structures}. A structure contains
\emph{fields} and \emph{embedded sub-structures}. A field contains a
value (i.e. a reference); an embedded sub-structure is an instance of
another type, which is created along with the encapsulating instance.
Though implementations may vary, the intended effect is that fields
appear in the memory layout in the order they appear in the structure,
and embedded sub-structures are really embedded, i.e. use for their own
memory layout the corresponding chunk of memory of the encapsulating
structure.

Each field has a type, which is a filter on possible values of the
field: these values must have types which are sub-types of the field
type. This is a side-effect; the primary function of the field type is
to qualify the declaration of accessor functions for that type. Embedded
structures also have a type, which defines which structure is embedded.
Only other structures may be embedded; it is not possible to embed basic
types (and it would not make much sense either). Embedding is acyclic: a
structure may not directly or indirectly embed itself.

Arrays of fields and arrays of embedded structures can be defined, with
a fixed number of elements.

Consider the following example. We suppose that the current namespace
is ``\verb|def|'', and that the ``\verb|bar|'' type is defined, or
to-be-defined, in that namespace.
\begin{verbatim}
    struct foo
        x int
        b1 bar
        b2 && bar
        c1 16 bar
        c2 && 16 bar
    end
\end{verbatim}

A structure named \verb|def::foo| is defined, with the following contents:
\begin{itemize}

    \item A field called \verb|def::x|, that may contain values of type
    \verb|std::int| or sub-types thereof (but there cannot be strict
    sub-types of \verb|std::int|). Note that the unqualified name
    ``\verb|int|'' is converted by an active alias to ``\verb|std::int|'',
    because that alias is part of the export list from the namespace
    \verb|std|, which is imported by default.

    \item A field called \verb|def::b1| for references to instances of
    \verb|def::bar| (or sub-types thereof).

    \item An embedded structure of type \verb|def::bar|, with name
    \verb|def::b2|.

    \item An embedded array of 16 references of type \verb|def::bar|,
    with name \verb|def::c1|.

    \item An embedded array of 16 embedded sub-structures of type 
    \verb|def::bar|, with name \verb|def::c2|.

\end{itemize}

No two elements of a structure may have the same name, regardless of
their respective kinds.

\subsubsection{Closing}

When first encountered, a structure type is created in an ``open''
state. This means that its name becomes known, but the full contents are
not yet defined. As long as a structure is open, new fields and embedded
elements (sub-structures, arrays, and arrays of embedded sub-structures)
can be added to the structure type. Once the type is \emph{closed}, no
new contents may be added. In the example above, \verb|def::foo| is
still open: new fields and embedded elements could still be added to the
structure. The ``\verb|end|'' keyword does not close the structure; it
merely exits the syntactic construction that is used to add elements to
a structure.

Similarly, the \verb|def::bar| structure, if not yet defined at this
point of the source code, has been automatically created, in open state
and with no initial contents, when the name ``\verb|def::bar|'' was
first encountered (i.e. when the field \verb|def::b1| was
defined).\footnote{Strictly speaking, the \texttt{\textbf{def::bar}}
structure, if created at this point, is in \emph{implicit} open state;
if it was not explicitly defined at the time \texttt{\textbf{def::foo}}
is closed, then an error is reported, under the assumption that a type
which was never explicitly defined anywhere is probably a typing
mistake.}

A structure will be closed in the following circumstances:
\begin{itemize}

    \item When closed explicitly with a specific function call on the
    type instance (i.e. the instance of type ``\verb|std::type|'' that
    represents this type).

    \item When an instance of the structure is created. Instance creation
    implies memory allocation, which needs the layout to be fixed.

    \item When an encapsulating structure is closed. For a structure to
    be closed, all the structures it embeds must first be closed. Since
    the embedding relationship is acyclic, this process converges.

    \item When the structure is made part of code being compiled.

\end{itemize}

Conversely, sub-types and super-types can be added to a closed
structure; that is, even after \verb|def::foo| is closed, new structures
can be defined and made sub-types of \verb|def::foo|, and
\verb|def::foo| itself can be made sub-types of other structures; these
additions are immediately inherited by existing instances of
\verb|def::foo|.

\subsubsection{Instantiation}

When a structure type \verb|T| is defined, a function with the same name
is created. That function takes no parameter, and returns an instance of
\verb|std::type| which represents the type \verb|T|.

A dedicated function \verb|std::new| takes as parameter a
\verb|std::type| instance, and creates a new instance of the represented
type. The fields of the new instance are set to uninitialized state
(except fields of boolean or modular integer types, which are set to
their default \verb|false| or zero values); this also applies,
recursively, to all embedded elements.

Calling \verb|std::new| on \verb|std::type| instances that do not
represent structure types triggers an exception.

\subsubsection{Accessors}

Structure contents can only be inspected and altered through dedicated
\emph{accessor functions}. These special functions are created when the
structure is closed. The accessors use the element names, depending on
the kind of element:
\begin{itemize}

    \item For a field of name \verb|def::x|, the functions \verb|def::x|
    and \verb|def::->x| are defined, to read and write values to the
    field \verb|def::x| of an instance, respectively. The accessor
    \verb|def::Z->x| clears the field, i.e. sets it to uninitialized
    state. The accessor \verb|def::x?| tests whether the field is
    initialized or not.

    \item For an embedded structure of name \verb|def::x|, one accessor
    function of name \verb|def::x&| is defined, which takes as input
    a reference to an instance of the encapsulating structure, and returns
    a reference to the instance embedded within it.

    \item For an array of references, with name \verb|def::x|, the
    functions \verb|def::x@| and \verb|def::->x@| read and write values
    into the array slot indexed by a given \verb|std::int| value.
    Also, \verb|def::Z->x@| clears a slot, and \verb|def::x@?| tests
    its initialization status. Finally, \verb|def::x*| initializes an
    array instance (of the right type) to provide an array view of the
    references.

    \item For an array of embedded sub-structures, with name \verb|def::x|,
    the \verb|def::x@&| accessor returns a reference to one of the
    embedded sub-structures, by \verb|std::int| index, and \verb|def::x*|
    initializes an array view of the sub-structures.

\end{itemize}

Fields, and slots in embedded arrays of references, are initially
uninitialized. There is no null value; reading an uninitialized field
triggers a runtime error.

For booleans and modular integers, i.e. the \verb|std::iXX| and
\verb|std::uXX| types, corresponding fields are always initialized.
Their starting value is \verb|false| (for booleans) or zero (for modular
integers), and the clearing accessors restore that value. The test
accessors (\verb|def::x?|, \verb|def::x@?|) then always return
\verb|true|. Note that plain \verb|std::int| fields are not in this
situation, and can be truly uninitialized.

\subsubsection{Extension}

A structure may \emph{extend} another structure; this is a combination
of sub-typing and embedding. When structure \verb|B| extends the
structure \verb|A|:
\begin{itemize}

    \item \verb|B| is defined to be a sub-type of \verb|A|.

    \item \verb|B| embeds an instance of \verb|A|, under the name of
    \verb|A| (that is, the accessor \verb|A&| is defined, that takes
    as input parameter a reference to an instance of \verb|B|, and
    returns a reference to the embedded instance of \verb|A|).

    \item Accessors for elements of \verb|A| can be used on an
    instance of \verb|B|, and will access the corresponding elements
    in the instance of \verb|A| which is embedded in \verb|B|.

\end{itemize}

Since extension is both embedding and sub-typing, it combines the
requirements of both; notably, extension cannot be done in a closed
structure, and the extension relationship is acyclic.

A given structure \verb|B| may directly extend a given structure
\verb|A| only once (this is implied by the fact that the extended
structure is embedded under its own name, and names are unique within
structure contents). However, a structure may extend several other
structures. This is a multiple inheritance model, which is powerful but
implies some ambiguous situations. Suppose, for instance, that:
\begin{itemize}

    \item Structure \verb|A| has a field named \verb|x|.

    \item Structure \verb|B| extends \verb|A|.

    \item Structure \verb|C| extends \verb|A|.

    \item Structure \verb|D| extends \verb|B| and \verb|C|.

\end{itemize}
In that situation, the \verb|x| function, which is the read accessor for
the field of the same name in \verb|A|, could be invoked on an instance
of \verb|B| and on an instance of \verb|C|. Since \verb|D| extends
\verb|B|, the accessors that accept an instance of \verb|B| will also
work on an instance of \verb|D|. However, the same can be said about the
accessors that accept an instance of \verb|C|. In fact, since \verb|D|
embeds both a \verb|B| and a \verb|C|, and each embeds an \verb|A|, the
structure \verb|D| indirectly embeds \emph{two} instances of \verb|A|,
and it is unclear which one is supposed to be used when reading the
field \verb|x|. Therefore, invoking \verb|x| on an instance of \verb|D|
triggers an error.\footnote{As we shall see, part of the work of the
compiler is to prove that such an error cannot happen in a given piece
of code.}

It is possible to make an explicit decision, by defining a function
called \verb|x|, attached to type \verb|D|, which then selects the
instance to use:
\begin{verbatim}
    : x (D) B& x ;
\end{verbatim}
This snippet reads as follows:
\begin{itemize}

    \item The ``\verb|:|'' token starts the definition of a new function.
    It is followed by the name of that function (\verb|x|).

    \item The ``\verb|(D)|'' expression registers the new function to
    the type ``\verb|D|''; that is, if a function call for name \verb|x|
    is encountered, and at that time the top element on the stack has
    type \verb|D|, then this function shall be called (and not, in
    particular, the accessor function which is registered on type
    \verb|A|).

    \item Afterwards follows the function body, which here consists in two
    successive function calls: \verb|B&|, which returns a reference to the
    sub-structure embedded in \verb|D| under the name \verb|B|, and then
    \verb|x|. Since that \verb|x| call will operate on the \verb|B|
    instance returned by \verb|B&|, it will use the \verb|A| instance
    embedded in that \verb|B| instance, and not the one embedded in the
    \verb|C| instance which is also embedded in \verb|D|.

    \item The semicolon token (``\verb|;|'') terminates the function body.

\end{itemize}
Thus, this new function explicitly chooses \verb|B|, not \verb|C|. Since
it is registered with type \verb|D|, which is a sub-type of \verb|A|, it
will have precedence over the \verb|x| accessor function defined on
\verb|A|, when invoked over an instance of \verb|D|.

\begin{rationale}
Type extension in T1 maps to the Java extension of classes, while
sub-typing corresponds to the Java extension of interfaces. Historically,
Java had only classes, and interfaces were added afterwards to compensate
for the lack of multiple inheritance. The Java inheritance has several
facets:
\begin{itemize}

    \item Inheritance of storage: state held in the superclass is also
    contained in the subclass instance. In T1, this is done with extension.

    \item Inheritance of behaviour: methods attached on the superclass
    also work with subclasses. This is what sub-typing provides in T1.

\end{itemize}
\end{rationale}

\subsection{Arrays}

Array types are defined on-demand. For a given type \verb|T|, the type
``array of \verb|T|'' has the name:
\begin{verbatim}
    (T std::array)
\end{verbatim}
(including the parentheses). This name is not a name token, as returned
by the lexer; however, as will be explained in the description of the
interpreter, the array type name mimics a sequence of code that, when
processed by the interpreter, yields a reference to the \verb|std::type|
instance that represents the array type. In fact, the \verb|std::array|
function \emph{creates} the array type if it does not already exist, and
registers all accessor functions for array instances.

Array instances really are \emph{views} on a chunk of memory. An
instance of the array type, when instantiated but not initialized,
points to nothing, and calling data accessors triggers an exception.
An array instance is populated in basically four ways:
\begin{itemize}

    \item Make the array instance point to a sequence of values or
    embedded structures within a given structure instance. If the
    sequence of values or embedded structures was declared with the name
    \verb|def::x|, then this is done with the \verb|def::x*| accessor
    function.

    \item Make the array instance point to a locally allocated sequence
    of values or embedded structures. For a local name \verb|x|, this is
    done by using the \verb|x*| pseudo-function name.

    \item Dynamically allocate a new memory chunk, with a specified
    length. When dynamic memory allocation is supported, this is done
    with the \verb|std::make| function.

    \item Initialize the array instance as a sub-array of another array
    instance. The sub-array must be entirely contained within the source
    array. This is done with the \verb|std::sub| function. An array
    instance can be reinitialized as a sub-array of itself with
    \verb|std::subself| (which is just a shorthand for \verb|std::sub|
    with the instance used for both operands).

\end{itemize}

There is thus always an indirection layer when accessing memory chunks.
Memory chunks themselves are not objects, i.e. they cannot be accessed
directly, and do not have a T1 type. Native code called from T1 can
obtain a direct pointer to the data, subject to some caveats (in
particular, objects may be moved in memory by the garbage collector, if
used; and locally allocated objects cease to exist when the owner
function returns): T1 memory safety guarantees that all array accesses
are ``safe'' (e.g. out-of-bounds accesses trigger a runtime error, and
all reachable objects are maintained in memory to avoid dangling
pointers), but native code can bypass such safety features.

\begin{rationale}
The Go language has \emph{arrays} and \emph{slices}. An array is a
sequence of value, and a slice is a view on such a sequence. Most
operations that work on arrays also work on slices. In T1, the Go arrays
become ``chunks of memory'' and are not directly accessible; the T1
``arrays'' are equivalent to the Go slices.

Thus, a T1 array type can be thought of as a structure with three
fields: pointer to the actual object that contains the data, offset and
length of the chunk within that hidden object. There is no notion of
``capacity'' as in Go (T1 arrays are not intrinsically growable).
\end{rationale}

An ``array of \verb|T|'' is an array of references (to elements of type
\verb|T|, or sub-types thereof). A newly created memory chunk will have
all slots uninitialized (or set to \verb|false| or zero, for booleans
and modular integers). The following accessor functions are defined:
\begin{itemize}

    \item \verb|std::make| dynamically allocates a new memory chunk,
    and initializes the array instance to point to that chunk.

    \item \verb|std::sub| initializes an array as a view of a chunk of
    another array. The source array must have been initialized.

    \item \verb|std::subself| merely duplicates the argument, then calls
    \verb|std::sub|.

    \item \verb|std::init?| returns \verb|true| if the array instance
    was initialized, \verb|false| otherwise. If the array instance was
    not initialized, then calls to the other functions below will
    trigger a runtime error.

    \item \verb|std::length| returns the length of the array (number of
    elements).

    \item \verb|std::@| and \verb|std::->@| read and write a value from
    an array slot or to an array slot, respectively, indexed by an
    \verb|std::int| value. Array indexes start at zero.

    \item \verb|std::Z->@| clears an array slot, and \verb|std::@?|
    returns its initialization status. For an array of booleans or
    modular integers, clearing a slot means setting it to \verb|false|
    or zero, and \verb|std::@?| always returns \verb|true|.

\end{itemize}

Types for arrays of embedded structures can also be obtained with
\verb|std::array&|. The expression:
\begin{verbatim}
    (T std::array&)
\end{verbatim}
will return an array type that, when instantiated and initialized, is
a view to a sequence of contiguously allocated instances of \verb|T|.
For such an array, the accessors that use or return references
(\verb|std::@|, \verb|std::->@|, \verb|std::Z->@| and \verb|std::@?|)
are not defined; instead, the accessor \verb|std::@&| returns a
reference to one of the structures embedded in the array.

Any other type can present an array-like interface by defining the
appropriate methods. It shall be noted that support for the
\verb|std::sub| function implies that any array-like type must
be able to present some of its contents in a contiguous sequence in
memory.

\begin{rationale}
The main idea behind arrays-as-views is to make it so that any function
that can work on arrays will also work on a sub-array. In practical Java
or C\# code that processes binary data (e.g. I/O code), several methods
are often needed, e.g. a \verb|write()| that takes as parameter an array
of bytes, and another \verb|write()| method that takes as parameters an
array of bytes, a start offset and a length. Array views are meant to
avoid these multiple methods. Moreover, they allow user-defined types
with array-like interfaces to be used instead (e.g. a growable vector of
bytes).
\end{rationale}

\subsection{Generics}

The on-demand creation of array types is an example of how generic types
are managed in T1. In full generality, container types are meant to be
created with metaprogramming. For instance, growable array types are
created with \verb|std::list|. The following sequence of code:
\begin{verbatim}
    (u8 list)
\end{verbatim}
will return an \verb|std::type| instance that represents growable arrays
of bytes. This is a normal structure type (albeit with a name that is
not a qualified name token), and the \verb|new| function on that type
instance will return a new growable array of bytes with an initial size
of zero. Each growable array type is a sub-type of the corresponding
array type, and offers the relevant accessor functions, as well as some
extra functions to append or remove elements.

Syntactic facilities are made available to users, in order to define
their own generic types. This is not restricted to making new types
parametrized by other types; this is more an expression that source code
can define functions and invoke them during the interpretation itself,
to process further source code and define other functions in arbitrary
ways.

\begin{rationale}
In Java, an important point of generics is that they impact the type
analysis, but do not create new types: \verb|ArrayList<String>| and
\verb|ArrayList<Date>| both use the same \verb|Class| instance (their
respective \verb|getClass()| methods return the same object), and cannot
be distinguished from each other at runtime. This is an historical
consequence of generics being added only relatively late in the language
(for Java 5). The generics are handled as an extra layer at compile-time
that is used to avoid having the developer make explicit casts and risk
the dreaded \verb|ClassCastException|.

Conversely, in C\#, \verb|List<string>| and \verb|List<DateTime>| are
distinct types with distinct runtime \verb|Type| instances (as returned
by \verb|GetType()|). The C\# compiler analyzes the source code to make
sure that, when replacing the type parameters with actual types (that
comply with the expressed restrictions on the type parameters), the
resulting code will still be valid; but the runtime machine will create
as many distinct types as necessary. T1 works similary to C\#, minus the
initial abstract analysis: in T1, we do not really care whether things
\emph{would} work with some large categories of types, but whether they
\emph{will} work with the types that the source code actually uses.
\end{rationale}

\newpage
\section{Functions}

Every piece of code in T1 is a \emph{function}. A function has a name
(which is a character string) and is \emph{registered}; the registration
is what makes the function callable. Several functions may have the same
name, but will then differ by the types under which they are registered.

\subsection{Runtime Model}

Function parameters, and returned values, are exchanged on a
\emph{stack}. The stack contains only values, which are references.
Every function extracts the parameters it needs from the stack, and
pushes back its returned values on the stack. This inherently allows
functions to return several values.

A function, when invoked, has an \emph{activation context}, which is
traditionally called a \emph{stack frame}. This is disjoint from the
stack described above. Implementations may use a stack structure to
allocate activation contexts; in that case, that stack structure is
often called the ``system stack'', while the normal stack for values is
called the ``data stack''. Here, we will strive the avoid the confusion
by using the expression ``activation context'', and reserving the term
``stack'' for the data stack.

The activation context is a transient memory area that will contain the
local variables for the function, and may save the current execution
point for the function. When a function calls another function, the
current instruction pointer is saved in the function activation context,
and a new activation context is created for the called function; when
that called function returns, its activation context is released, and
the instruction pointer is restored from the activation context of the
caller. Exactly how this happens is an implementation detail.

Local variables are slots that can receive values; during translation of
the source code, local variables have names, which allows source code to
issue read and write instructions for these variables.

\begin{rationale}
In Forth, the system stack is explicit, with words \verb|R>| and
\verb|>R| to move values from the system to the data stack, and back.
Since some tasks may require more complicated data movements (the usual
example is adding tridimensional vectors together), Forth also defines
local variables, which are usually located on the system stack. Local
variable names are translated at compilation time into depths on the
system stack, which means that local variables don't interact well, or
at all, with facilities that access the system stack, such as explicit
words (\verb|R>| and its ilk...), or loop counters. Thus, a function
may use the system stack explicitly, \emph{or} use local variables,
but should not try to mix both.

For T1, which is not encumbered by compatibility with existing legacy
code, it seems simpler to avoid the complications and normalize on a
single system. Thus, local variables have been chosen, and the system
stack is not made visible to user code, except as the ``activation
context'' abstraction.
\end{rationale}

\emph{Locally allocated instances} are an extension of local variables:
these are object instances that are part of the activation context. This
corresponds to automatic variables in C or C++; arrays of references or
embedded structures can be obtained that way. These locally allocated
instances are nominally destroyed when the owner function exits. T1 does
not have destructors (in the C++ sense) or finalizers (in the Java
sense), thus the notion of ``destruction'' really means memory
deallocation. During interpretation, the garbage collector is used for
such instances, meaning that local allocation is not different from
normal heap allocation. In compiled code, allocation is really done in
the activation context, and has some restrictions so that memory safety
is maintained in all its facets (notably guaranteed maximum stack
growth): the size of such instances must be known at compile-time, and
instances shall not ``escape'' to outer contexts, i.e. remain reachable
once the owner function has returned.

\begin{rationale}
A typical use for stack allocation is creation of an array view instance
to designate a chunk of an array provided by a calling function, for
purposes of using that new array view instance as parameter to another
nested function. Such operations should be doable even when compiling
for targets that do not support dynamic memory allocation. Another use
is assembly of a small character string, e.g. for immediate display.
\end{rationale}

\subsection{Function Invocation}

The only way to invoke a function is by name. Function names may be
arbitrary; syntactically, a name token is used, and unqualified names
are translated to qualified names by the parser, thus most function
calls should use qualified names.

To be callable, a function must be \emph{registered}. A function
registration includes its name, and parameter types. For instance,
this code defines and registers a function:
\begin{verbatim}
    : foo (int string)
        # Here goes the function body
\end{verbatim}
If the current namespace is \verb|def|, then the function is registered
under the name \verb|def::foo| and with two parameter types,
\verb|std::int| and \verb|std::string|. The intent is that if some code
calls the function ``\verb|def::foo|'', and at that exact time, the
runtime types of the top two stack elements are \verb|std::int| and
\verb|std:string|, respectively, then the function defined above shall
be the one to be called. Types are provided in ``stack order'', i.e.
the rightmost element is the top-of-stack.

The function invocation process works in the following way:
\begin{itemize}

    \item The function invocation uses a specific name; only functions
    registered under that exact name are considered.

    \item A set of all \emph{matching functions} is defined: these are
    all the functions (with the correct invocation name) for which the
    registered parameter types match the runtime types of the
    corresponding stack elements at call time. E.g. in the example
    above, that function \verb|def::foo| is a matching function if and
    only if the top stack element has type \verb|std::string| or a
    sub-type thereof, and the stack element immediately below has type
    \verb|std::int| or a sub-type thereof.\footnote{In that specific
    case, \texttt{\textbf{std::int}} cannot have sub-types, but
    \texttt{\textbf{std::string}} can.}

    \item The matching functions are ordered by \emph{precision}. Let
    $f$ be a function registered with parameter types $r_m, r_{m-1},...
    r_1$ (in stack order, $r_1$ is top-of-stack), and $g$ be a function
    registered with parameter types $s_n, s_{n-1},... s_1$. $f$ will be
    said to be \emph{more precise} than $g$ if and only if all of the
    following hold:
    \begin{itemize}

        \item $m \ge n$ (i.e. $f$ is registered with at least as many
        parameter types as $g$)

        \item For all $1\le i\le n$, type $r_i$ is a sub-type of $s_i$.

    \end{itemize}

    \item If one of the matching functions is more precise than all other
    matching functions, then that function is called. Otherwise, an
    error occurs.

\end{itemize}

The following important points must be noticed:
\begin{itemize}

    \item Precision order is partial. Two given functions are not
    necessarily comparable, i.e. neither being ``more precise'' than
    the other. The invocation process does not require that all matching
    functions be comparable to each other, but that one can be compared
    to all others, and found to be more precise than all others.

    \item A failure will be reported if there is no matching function,
    but also if there are several and none is more precise than all the
    others.

    \item Since sub-typing is acyclic (except that every type is deemed
    to be a sub-type of itself), the only way for two functions $f$ and
    $g$ to be such that $f$ is more precise than $g$ and $g$ is more
    precise than $f$ at the same time, is to have $f$ and $g$ registered
    with the exact same parameter types. This situation is explicitly
    forbidden: any attempt at registering a function with the same name
    and parameter types as an already registered function triggers an
    error.

    \item If a function is registered with $n$ parameter types, and the
    stack contains fewer than $n$ elements at call time, then that
    function is not a matching function.

    \item The parameter types used for registration do not necessarily
    exhaust all the actual function parameters. A function registered
    with two parameter types may use more than two parameters. Moreover,
    registration says nothing about how the number and types of values a
    function may return (i.e. leave on the stack when exiting).

\end{itemize}

This process works best when registered functions use the same patterns.
For instance, it is expected that most functions in a given application
will work like ordinary methods as in classic OOP, i.e. be dispatched
based on the type of a single parameter, which will be ``the object on
which the method is invoked''. To allow functions to be used without
undue collisions, even if the same names are used, it is best if all
such method-like functions are registered such that the owner object is
the top-of-stack (i.e. rightmost parameter in the list).

\subsection{Immediate Functions}

An \emph{immediate function} is a function which is registered with no
parameter types, and a special ``immediate'' flag. The role of immediate
functions is to be invoked as soon as they are encountered in the source
code, during interpretation; this is how additional syntax is defined.

\newpage
\section{Interpretation Syntax}

\emph{Interpretation} is the process during which source code is
translated into instructions to execute. The source code may itself
trigger the immediate execution of the functions which it just defined;
they then run in the context of the interpreter. \emph{Compilation} is a
separate step that may optionally occur when triggered by the
interpreted code, or implicitly at the end of interpretation, or not at
all; this is covered in a later section.

The source code syntax is defined in terms of interpreter behaviour.
Like in Forth, there is no formal syntax that is parsed into a tree;
instead, the main interpreter is a simple loop, which is then extended
by \emph{immediate functions}, which are functions that are executed
immediately when their name is encountered, and interact with the source
stream to implement all extra syntax. User code can define its own
immediate functions, and thereby process the source code in arbitrarily
extensible ways.

\subsection{Function Building}

At any point, the interpreter is \emph{building a function}, i.e.
accumulating instructions into an as-yet incomplete function. Building
contexts nest: at any time, a new building context may be opened; the
previous one will be restored when the new context is closed. Closing
the context yields the function.

Some building contexts are said to be \emph{automatic}: whenever some
instructions have been accumulated in an automatic context and there is
no outstanding flow control structure, the context is closed (which
creates the corresponding function), cleared, and reopened; the function
which was just created is then executed.

For a non-automatic building context, the new function is registered as
soon as it is created. For an automatic building context, the new
function is not registered and does not have a name; it is executed
right away, then discarded. Note that the automatic context is cleared
and reopened \emph{before} running the newly created function: this
allows that function to populate the building context with new function
elements.

\begin{rationale}
In Forth, the interactive system has two states, ``interpretation'' and
``compilation''. In interpretation mode, typed words are executed
immediately, while in compilation mode, they are recorded in the
currently-built function (``word'', in Forth terminology). The
interpretation/compilation duality complicates the description of the
language, in that many words have different semantics depending on the
current state. In plain Forth (not counting some non-standard
extensions), states do not stack, and you cannot define a sub-function
within a function. Moreover, flow control structures are not available
in the interpreter.

In T1, the term ``compilation'' is reserved for a distinct process,
described later on. Thus, ``interpretation'' is used for all activities
related to source code processing. The automatic building contexts are
functionally equivalent to the Forth ``interpreter'', except that they
allow all flow control structures, and do not require special semantics.
\end{rationale}

Functions are made of the following formal instructions:
\begin{itemize}

    \item \textsc{call}: invoke a function with a specific name
    (normally a qualified name token).

    \item \textsc{const}: push on the stack a given value (a reference).

    \item \textsc{getlocal}: push on the stack the value currently held in
    a specified local variable.

    \item \textsc{getlocalindex}: push on the stack the value currently held
    in a specified local variable (among an array of locals, by index).

    \item \textsc{putlocal}: pop a value from the stack and write it into
    a specified local variable.

    \item \textsc{putlocalindex}: pop a value from the stack and write
    it into a specified local variable (among an array of locals, by
    index).

    \item \textsc{reflocal}: push on the stack a reference to a
    locally allocated instance.

    \item \textsc{reflocalindex}: push on the stack a reference to a
    locally allocated instance (among an array of locally allocated
    instances, by index).

    \item \textsc{ret}: exit the current function, returning control to
    the caller.

    \item \textsc{jump}: unconditional jump to another point in the
    sequence of instructions (within the same function).

    \item \textsc{jumpif}: conditional jump to another point in the
    sequence of instructions: the top-of-stack is popped and must be
    a boolean value; the jump is taken if that value is \verb|true|.

    \item \textsc{jumpifnot}: conditional jump to another point in the
    sequence of instructions: the top-of-stack is popped and must be
    a boolean value; the jump is taken if that value is \verb|false|.

\end{itemize}

Local elements are statically indexed, i.e. which local variable or
instance, within the current activation context, is used in a
\textsc{putlocal}, \textsc{getlocal} or \textsc{reflocal}, is a question
which is decided at the time the instruction is added to the currently
built function. For \textsc{getlocalindex}, \textsc{putlocalindex} and
\textsc{reflocalindex}, the location and length of the sequence of local
variables or instances are also decided at function building time; the
index is popped from the stack at runtime, and compared with these
bounds to prevent out-of-bounds accesses.

The \textsc{jump}, \textsc{jumpif} and \textsc{jumpifnot} opcodes are
together called the ``jump opcodes''.

\subsection{The Interpreter Loop}\label{syntax:loop}

The interpreter loop is described in pseudo-code as follows:
\begin{enumerate}

    \item \label{interp:start}Read the next token from the source code
    stream. If there is no next token (end of source stream), exit (the
    interpretation process is finished).

    \item \label{interp:const}If the token is a numerical constant or a
    literal string, then add a \textsc{const} opcode to the current
    function for that value (for character strings, this implies
    creating the instance that contains that string, and using the
    reference to that new instance as value); then jump to
    step~\ref{interp:auto}.

    \item \label{interp:name}The token is a name. If the name is
    unqualified:
    \begin{enumerate}

        \item If the name matches that of an accessor for a local
        variable or locally allocated instance, then the corresponding
        opcode (\textsc{getlocal}, \textsc{putlocal}...) is added to the
        current function; then jump to step~\ref{interp:auto}.

        \item If there is a currently defined alias for that name,
        then convert the name into the qualified name to which the
        alias points.

        \item Otherwise, convert the name to a qualified name by
        adjoining the current namespace.

    \end{enumerate}

    \item \label{interp:immediate}The name is qualified. If there
    is a currently registered immediate function under that name,
    invoke it immediately, then jump to step ~\ref{interp:auto}.

    \item \label{interp:call}Add a \textsc{call} opcode to the
    current function, for the qualified name.

    \item \label{interp:auto}While all of the following hold:
    \begin{itemize}

        \item the current building context is automatic;

        \item the current building context is not empty;

        \item the current context does not have outsanding flow
        control structures;

    \end{itemize}
    finalize the current context into a function $f$, reinitialize the
    context into a new empty function builder, and execute the function
    $f$.

    \item \label{interp:loop}Jump to step~\ref{interp:start}.

\end{enumerate}

For this description, in step~\ref{interp:start}, we assume that the
source code is a single input stream. In practice, the interpreter will
handle several successive source files, one at a time, with a new
interpreter loop for each file. Building contexts are not conserved
across files, so any function whose building has started must be
finished by the end of the same file.

Numerical constants are the two boolean values (\verb|true| and
\verb|false|), character constants, and number constants. Character
constants are words that start with a backquote character
(``\verb|`|''), while number constants start with an ASCII digit, or a
plus or minus sign followed by an ASCII digit. Literal strings start
with a double-quote character (``\verb|"|''). The syntax for numerical
constants and literal strings was described in section~\ref{lexing}.

\begin{rationale}
\textsf{\textbf{TODO:}} The constant parsing process will be made
pluggable, so that new arbitrary constant formats may be defined,
normally distinguished by suffix. For instance, when ``big integers''
are implemented, they will use numerical constants with a ``\verb|z|''
suffix. Similarly, floating-point constants will use a dot symbol
(``\verb|.|'') and the usual exponent notation. In all generality, there
will be a sequence of registered functions that are invoked in due order
until one returns that it could understand the format; the last one will
apply the rules for integer types.
\end{rationale}

In step~\ref{interp:name}, a \textsc{putlocal} into local variable
\verb|x| is obtained with the name ``\verb|->x|'', without a space
between ``\verb|->|'' and ``\verb|x|''. If a space separates both parts,
then a \textsc{putlocal} will also be obtained through a much different
road, the name ``\verb|std::->|'' being itself an immediate function
that implements an extended syntax for writing into local variables.

In step~\ref{interp:immediate}, the ``immediate'' flag is an extra
information attached to the function when registered. A normal
\textsc{call} opcode that targets an immediate function name (i.e.
adding a call to the immediate function in the currently built function,
rather than calling the immediate function immediately) can be obtained
with the ``quoting function'' (\verb|'|, described later).

In step~\ref{interp:auto}, a loop is used because execution of the
current function may again add opcodes to the current automatic context.
Note that the context is cleared and reinitialized after finalizing the
current function, but before calling it, precisely so that new opcodes
may be added to the context without being discarded. The call to the
built function is direct and does not use the name and lookup process
(the function is not actually registered).

The meaning of ``outstanding flow control'' will be explained in
section~\ref{syntax:flow}.

\subsection{Nested Interpreter}\label{syntax:nested}

While a given interpreter loop can work over nested building contexts,
a common pattern in the T1 syntactic constructions is the use of a
nested interpreter loop. This is normally triggered by the opening
parenthesis token (``\verb|(|''):
\begin{itemize}

    \item A new builder context is created and opened. This is an
    automatic context, i.e. with immediate execution of instructions.

    \item A new, empty data stack is created.

    \item The nested loop runs until it reaches a closing parenthesis
    token (``\verb|)|''), using the new data stack. If the end of the
    source stream is reached before obtaining that closing parenthesis,
    an error is raised.

    \item When the nested loop exits, the caller obtains the contents of
    the data stack which was created for the nested loop. If, at that
    point, the current builder context is not the automatic context that
    was created for the nested interpreter, or that context is not empty
    (because of an outstanding flow control structure), then an error is
    raised. Otherwise, that context is removed, and the previous
    context, which was active when the opening parenthesis was
    encountered, is restored.

\end{itemize}

One case is when the interpreter loop encounters the opening paranthesis
in its normal processing loop, at step~\ref{interp:name} (specifically,
when after applying aliases, the qualified name is ``\verb|std::(|'').
In that case, a nested interpreter loop is launched, in the conditions
described above. When that loop exits, the contents of its dedicated
data stack are used for that many \textsc{const} opcodes added to the
current builder context.

Other uses of a similar construction are for function and type
declarations.

\subsection{Function Declaration}

To declare (and define) a new function, the ``\verb|:|'' function is
used. Here is an example:
\begin{verbatim}
    : fact <export> (u64)
        # ...
\end{verbatim}
The behaviour of ``\verb|:|'' is as follows:
\begin{enumerate}

    \item Get the next token from the source stream; if it is a literal
    string, then the string contents are the name under which the
    function shall be registered; otherwise, it shall be a name token.
    In the latter case, if the name is unqualified, then it is converted
    to a qualified name:
    \begin{itemize}

        \item If there is a defined alias for the name, then the
        qualified name to which the alias points is used.

        \item Otherwise, a qualified name is made by adjoining the current
        namespace to the parsed raw name.

    \end{itemize}

    \item After the name may follow one or several of the following
    qualifiers:
    \begin{itemize}

        \item ``\verb|<export>|'': the new function will be added to the
        export list of the current namespace (see the
        section~\ref{syntax:namespace} for details).

        \item ``\verb|<immediate>|'': the new function will be registered
        as immediate when finished building.

    \end{itemize}
    Order of appearance is not significant for qualifiers; if the same
    qualifier is applied several times, this has the same effect as
    a single instance of the qualifier. Unknown qualifiers trigger an
    error.

    \item An opening parenthesis (``\verb|(|'') terminates the list
    of qualifiers, and starts a new, nested, automatic building context,
    as described in section~\ref{syntax:nested}. The stack contents upon
    exit of the nested interpreter loop are then used as the list of
    parameter types for registration of the new function. If any of the
    values is not a type (an instance of ``\verb|std::type|''), then
    an error is raised.

    If the new function is immediate, then the list of types shall be
    empty (immediate functions are registered with an empty parameter
    list); otherwise, an error is raised.

    \item A new building context is created for the new function. The
    function name (qualified), flags (exported, immediate...), and
    parameter types are stored in that context, and will be used when
    the context is closed.

\end{enumerate}

Since a new building context was created, subsequent actions of the
interpreter loop will add opcodes to that builder, hence contributing to
the code of that new function. The new function is not registered until
its building context is finalized. This is normally triggered by the
immediate function ``\verb|;|''.

The ``\verb|:|'' function is immediate, thus allowing the declaration of
a new function while another function is being built (this contrasts
with Forth, where nested function declarations are not supported, and
``\verb|:|'' is not immediate, since it is supposed to be invoked only
from the interpreter). These nested functions do not have any scoping
hierarchy or similar features: a nested function has the same visibility
as any other, and it cannot access the local variables and instances of
the outer function. This feature is mostly a syntactic convenience.

\subsection{Flow Control}\label{syntax:flow}

The jump opcodes are added with dedicated immediate functions, that
use a \emph{control-flow stack} which is managed by the builder context.
That stack is separate from the data stack. It contains ``origins''
and ``destinations'':
\begin{itemize}

    \item An \emph{origin} represents a jump opcode that has been added
    to the current builder, but still needs to be resolved to its
    destination.

    \item A \emph{destination} represents an opcode of any type that may
    become the target of a jump opcode.

\end{itemize}

\begin{rationale}
The control-flow stack is a powerful concept imported from Forth. In
Forth, it is implementation-dependent whether the control-flow stack
uses the data stack, or is separate; in T1, the control-flow is
separate and bound to the builder context, which avoids any issue with
nested function builders.
\end{rationale}

The concept behind the control-flow stack is that a jump opcode is
first added, then resolved; resolution occurs either when the target
destination becomes known, for a \emph{forward jump} (the target is
beyond the jump, and thus added later on), or when the jump opcode
itself is added, for a \emph{backward jump} (the target is before the
jump, and already present at the time the jump opcode is added):
\begin{itemize}

    \item An origin is pushed when adding a forward jump.

    \item An origin is consumed when adding the target for a forward
    jump.

    \item A destination is pushed when adding the target for a backward
    jump.

    \item A destination is consumed when adding a backward jump.

\end{itemize}
At any time, the builder has a \emph{current address}, which designates
the next opcode that will be added. Thus, whenever an origin or
destination is pushed, it designates the opcode that will be added next.

Consider for instance the classic ``if'' construction. As per the Forth
tradition, it syntactically looks as follows:
\begin{verbatim}
    ... # some code that pushes a boolean value
    if
        ... # executed if the boolean is true
    else
        ... # executed if the boolean is false
    then
\end{verbatim}
This code has two forward jumps:
\begin{itemize}

    \item a forward \textsc{jumpifnot} at the position of the
    ``\verb|if|'', to consume the boolean value and skip the first code
    chunk if the boolean falue is \verb|false|; that jump targets the
    second code chunk, just after the ``\verb|else|'';

    \item a forward \textsc{jump} opcode at the position of the
    ``\verb|else|'', so that after execution of the first code chunk
    (when the boolean was \verb|true| and the \textsc{jumpifnot} was not
    taken), execution skips to the code that follows the final
    ``\verb|then|''.

\end{itemize}

The behaviour of the three immediate functions is as follows:
\begin{itemize}

    \item ``\verb|if|'': push the current address as an origin, and
    add a \textsc{jumpifnot} opcode (thus, the ``origin'' qualifies
    that newly added opcode).

    \item ``\verb|else|'': push the current address as an origin,
    add a \textsc{jump} opcode, swap the two top elements of the
    control-flow stack, and pop the top element (it should be an
    origin) to resolve it against the current address.

    \item ``\verb|then|'': pop the top control-flow stack element
    (it should be an origin) to resolve it against the current address.

\end{itemize}
Thus, the ``\verb|if|'' adds an as-yet-unresolved forward jump, which
is pushed as an origin on the stack; ``\verb|else|'' adds another
forward jump, also pushed as an origin on the stack, and resolves the
first jump to the opcode that will immediately follow the second
forward jump; ``\verb|then|'' resolves the second jump opcode.

Since origins and destinations are organized as a stack, this naturally
supports nesting flow structures.

A builder is said to have \emph{outstanding flow control structures}
when its control-flow stack is not empty. In such a case, an automatic
builder does not finalizes itself.

\begin{rationale}
The use of the control-flow stack to decide whether an automatic builder
context finalizes and executes the current function or not, allows the
use of flow control structures in code meant for immediate execution.
This contrasts with Forth, where (normally) you cannot use flow control
in ``interpreter mode''.
\end{rationale}

When a function builder is finalized, its control-flow stack must be
empty, otherwise an error is raised. Also, all jump opcodes must have
been resolved. Normally, resolution is done by consuming items on the
control-flow stack; however, since items on the control-flow stack can
be explicitly duplicated and dropped, the two conditions ``stack is
empty'' and ``all jumps are resolved'' are not necessarily synonymous.

All functions end with an implicit \textsc{ret} opcode. Thus, if the
current address was used to resolve a forward jump, but no opcode was
added afterwards, the jump targets that implicit \textsc{ret}.

\subsection{Type Declarations}\label{syntax:types}

A structure type is defined with the \verb|std::struct| immediate
function. This function parses the new structure name, and then
the structure fields and embedded sub-structures. Here is an
example of such a declaration:
\begin{verbatim}
    struct foo <export>
        x   int              # reference field of type std::int
        vx  12 u8            # embedded array of 12 std::u8 values
        p   bar              # reference field of type def::bar
        q   && bar           # embedded structure of type def::bar
            && qux           # def::foo extends def::qux
        a   (i32 array)      # reference field of type (std::i32 std::array)
        b   && 5 (int list)  # embedded array of 5 instances of (std::int std::list)
    end
\end{verbatim}
The line breaks and indentation are not significant, and have been set
for clarity of the source code only. The comments (starting with
``\verb|#|'') are similarly not significant.

The example above illustrates the characteristics of the type declaration
syntax:
\begin{itemize}

    \item Elements are declared with the element name, followed by its
    type.

    \item If an integer constant lies between the element name and the
    type, then the element is an embedded array.

    \item The special name ``\verb|&&|'' is used to embed sub-structures.
    It can be combined with an integer for an embedded array of embedded
    sub-structures.

    \item If the element name is missing, and the ``\verb|&&|'' special
    name appears where a name was expected, then this is a type extension,
    which combines embedding and sub-typing. The name of the embedded
    class is also used as the element name.

    \item When an element type is expected, an opening parenthesis can be
    used, to create a nested interpreter loop that evaluates to the
    \verb|std::type| instance to use.

\end{itemize}

The behaviour of ``\verb|std::struct|'' is as follows:
\begin{enumerate}

    \item Parse the next token from the stream. If it is a literal
    string, then the string contents are the type name; otherwise, the
    next token shall be a name. In the latter case, if the name is not
    qualified, then it is converted to a qualified name by applying the
    currently defined aliases, or adding the current namespace if none
    of the current aliases applies.

    \item Get the type that currently has the specified name. If there
    is no such type, a new structure type is declared and used. If the
    type exists but is closed, then an error is raised; otherwise, the
    existing type will be used.

    \item If the next token from the stream is the name
    ``\verb|<export>|'', then the type function (the function that
    returns the ``\verb|std::type|'' instance corresponding to the new
    type) will be marked as exported (see section~\ref{syntax:namespace}
    for details); otherwise, the next token is pushed back onto the
    stream, to be read again at the next step.

    \item \label{struct:loop}Parse the next token $t$ from the stream. If
    that token $t$ is the name ``\verb|end|'', then the type declaration
    stops, and the immediate function ``\verb|std::struct|'' returns.
    Note that the type is \emph{not} closed.

    \item If the token $t$ is the name ``\verb|&&|'', then this is
    an extension:
    \begin{enumerate}

        \item A type reference is parsed. This must be a single type
        instance $T$, with no integer count. Type reference parsing is
        described later.

        \item A new element is added to the current structure, using the
        type $T$, with the name of $T$ as element name. Then go to
        step~\ref{struct:loop}.

    \end{enumerate}

    \item The token $t$ must be a name. If that name is unqualified,
    then it is converted to a qualified name $n$ by using the current
    aliases (if applicable), or the current namespace. Otherwise, $n$ is
    set to be equal to $t$.

    \item If the next token is the name ``\verb|&&|'', then that token
    is parsed (i.e. discarded from the input stream), and the new
    element will be an embedded sub-structure, or an embedded array of
    embedded sub-structures; otherwise, the next token is left on the
    input stream for the next step, and the element will be a field or
    an embedded array of fields.

    \item A type reference is parsed. This may be either a type instance
    $T$, or a pair consisting of an integer value $x$ followed by a type
    instance $T$. In the latter case, an embedded array of $x$ elements
    is defined; the value $x$ must be greater than zero (otherwise, an
    error is raised). The type $T$ applies to the new element (as type
    of the field, or the embedded structure, or the embedded array
    element values or embedded structures, depending on the presence of
    the integer $x$ and the initial ``\verb|&&|'' token).

    \item Go to step~\ref{struct:loop}.

\end{enumerate}

\emph{Parsing a type reference} is a sub-process that behaves as
follows:
\begin{enumerate}

    \item Start with an empty list of values. ``Adding to the list'' means
    appending a new value at the end of the list.

    \item \label{parsetype:loop}Get the next token $t$. If that token
    is a numerical constant, then it shall be of type ``\verb|std::int|''
    (otherwise, an error is raised); that value is added to the list,
    then the process loops to step~\ref{parsetype:loop}.

    \item If $t$ is an opening parenthesis (``\verb|(|''), then a
    nested interpreter loop is executed, as specified in
    section~\ref{syntax:nested}; the output contents of the data stack
    of that loop are then examined:
    \begin{itemize}

        \item If the nested stack contains only \verb|std::int| values,
        then these values are added to the list in stack order
        (top-of-stack is added last); then loop to
        step~\ref{parsetype:loop}.

        \item If the nested stack contains zero, one or more
        \verb|std::int| values, followed by a single \verb|std::type|
        instance (as the top-of-stack), then these values are added to
        the list in stack order; then jump to step~\ref{parsetype:exit}.

        \item Otherwise, the stack contents are not valid, and an
        error is raised.

    \end{itemize}

    \item If $t$ is a literal string, then the string contents are used
    as type name $n$. Otherwise, $t$ shall be a name; that name is used
    for $n$ (converted to a qualified name with the current aliases and
    namespace, if necessary).

    \item The type of name $n$ is added to the list. If that type does
    not exist, then a new empty, open structure of name $n$ is created,
    and its \verb|std::type| instance is used.

    \item \label{parsetype:exit}The list contents are returned.

\end{enumerate}

By construction, the parsing of a type reference can only return a
\verb|std::type| instance, preceded by zero, one or more \verb|std::int|
values.

\begin{rationale}
\textsf{\textbf{TODO:}} Allow multi-dimensional arrays. The type parsing
mechanism can return more than one integer value. It is unclear whether
multidimensional arrays are really a good idea, though: they are merely
a syntactic shortcut for computing the index as a multiplication and an
addition, since all dimensions are fixed (no ``jagged arrays''). Defining
multi-dimensional arrays would require making special accessor names,
e.g. ``\verb|v@@|'' to make it syntactically explicit that two index
values are expected.
\end{rationale}

\begin{rationale}
The type parsing mechanism allows the use of generics. Consider the
two following element declarations, which have similar effects:
\begin{verbatim}
    x  "(std::u8 std::list)"
    y  (u8 list)
\end{verbatim}
In the first case (element \verb|x|), the explicit name of the
``growable array of bytes'' type is used, while in the second case
(element \verb|y|), a nested interpreter loop is used; that loop will
first call the \verb|std::u8| function (which pushes the
\verb|std::type| instance of unsigned integers modulo $2^8$), then call
the \verb|std::list| function, which will use the \verb|std::type|
instance on the stack as parameter for creating the \verb|std::type|
instance for the growable array of bytes.

The second syntax is easier to use, because the nested interpreter loop
mechanics will include the automatic qualification (``\verb|u8|'' is
converted to ``\verb|std::u8|'' as per the aliases imported from
namespace \verb|std|) and be lenient about whitespace, whereas the use
of the literal string for \verb|x| requires using the exact type name.

Moreover, using the nested interpreter loop is also more robust: the
call to \verb|std::list| creates the type on demand, and, in particular,
also creates and registers all functions that operate on growable
vectors of bytes. The syntax with a literal string does not perform
this task, and thus relies on other constructions in the source code to
ensure that the said functions exist.
\end{rationale}

\subsection{Local Variables And Instances}

\emph{Local variables} are slots that can receive a value (i.e. a
reference) and that exist within the activation context of a function;
they disappear when the function exits. Similarly, \emph{local instances}
are object instances that are allocated when a function activation
context is created, and meant to be released when the function exits.

The generic syntax for creating local variables and instances mimics
that of the declaration of types. It starts with the \verb|local|
immediate function:
\begin{verbatim}
    local
        x   int              # reference field of type std::int
        vx  12 u8            # embedded array of 12 std::u8 values
        p   bar              # reference field of type def::bar
        q   && bar           # embedded structure of type def::bar
        a   (i32 array)      # reference field of type (std::i32 std::array)
        b   && 5 (int list)  # embedded array of 5 instances of (std::int std::list)
    end
\end{verbatim}

For each named element, \emph{accessor names} are created. These names
are recognized by the interpreter loop (see section~\ref{syntax:loop})
when building the function, and converted to the appropriate opcodes.
Such names do not exist beyond function building, are not bound to any
namespace, and cannot be aliases (they are matched before applying
aliases and namespaces). These names are the following:
\begin{itemize}

    \item For a field of name \verb|x|:
    \begin{itemize}

        \item \verb|x| returns the current contents of the field.

        \item \verb|->x| writes a value into the field.

    \end{itemize}

    \item For an embedded structure of name \verb|x|:
    \begin{itemize}

        \item \verb|x&| returns a reference to the structure.

    \end{itemize}

    \item For an embedded array of references of name \verb|x|:
    \begin{itemize}

        \item \verb|x@| reads a reference value, using an index (of type
        \verb|std::int|).

        \item \verb|->x@| writes a reference value, using an index (of type
        \verb|std::int|).

        \item \verb|x*| initializes an array instance (of the right type)
        to provide an array view of the array.

    \end{itemize}

    \item For an embedded array of embedded structures of name \verb|x|:
    \begin{itemize}

        \item \verb|x@&| returns a reference to one of the embedded
        structures, using an index (of type \verb|std::int|).

        \item \verb|x*| initializes an array instance (of the right type)
        to provide an array view of the array.

    \end{itemize}

\end{itemize}

Several \verb|local| declarations may exist within a function, provided
that no local name is reused within that function.

There is no smaller scope than a function. When a local variable or
instance is declared, its name becomes usable until the end of the
function building, but the corresponding variable or instance is created
when the function activation context is created, i.e. upon function
entry.

Local variables and instances are accessible only within the function in
which they were declared. In particular, if a new function builder is
opened without closing the current builder, the syntactically nested
function builder is separated from the outer function; it does not have
any access to the outer function's local variables and instances, and it
may create its own local variables and instances without any restriction
on types and names. The ``nesting'' feature does not have any
significance for the built functions.

Since accessor names are resolved syntactically, the types associated
with local variables are not used for any registration mechanism. They
are still used as filters for write accesses: if a field is declared
with type \verb|std::foo|, then only references to such a type, or a
sub-type thereof, may be written into the field. This is meant as a way
to document intended types for local variable contents. Compliance with
such type filters is verified dynamically by the interpreter (upon each
actual write access), and statically by the compiler.

\begin{rationale}
When a function is entered, local variables, and local instance fields,
are filled with their default values: booleans are \verb|false|, small
modular integers are zero, and all other types (including
\verb|std::int| fields) are uninitialized. Reading an uninitialized
field triggers an exception. Contrary to structures, no accessor names
are provided to test a local variable for initialization, or to reset it
to uninitialized state: the intent of local variables is to never be
read while still uninitialized, and the compiler will refuse to compile
functions for which it cannot prove that local variables are never read
before being written. This is meant to allow for more optimized usage of
local variables, without tests for uninitialized state, at least in
compiled functions. This also mimics the behaviour of both Java and C\#
compilers.
\end{rationale}

Since references to local instances can be obtained, it is possible to
access such instances after the activation context in which they were
created has been destroyed. This is permitted in the interpreter (which
implies that such local instances may actually be heap-allocated).
However, the compiler enforces escape analysis to make sure that such
survival does not happen, allowing local instances to be truly allocated
within the activation context.

An alternate syntax for declaring local variables uses the immediate
function \verb|std::{| (note that the opening brace ``\verb|{|'' is a
special character for the lexer, and thus a name by itself). The
folowing:
\begin{verbatim}
    {a b c}
\end{verbatim}
declares three local variables of names ``\verb|a|'', ``\verb|b|'' and
``\verb|c|'', respectively. They have type ``\verb|std::object|'', i.e.
can accept any value (reference), and are initially uninitialized.

The special immediate function ``\verb|std::->|'' can be used to write
to several local variables at once, or even to combine declaration and
initialization. When that function is executed (i.e. when encountered in
source code, since it is immediate):
\begin{itemize}

    \item If the next token is ``\verb|{|'', then this opens a list of
    names, ending with the closing token ``\verb|}|''. This both declares
    and initializes local variables.

    \item Otherwise, if the next token is ``\verb|[|'', then this opens
    a list of names, ending with the closing token ``\verb|]|''. This
    writes to several local variables, but does not declare them; the
    local variables must already exist.

    \item Otherwise, the next token must be the name of an already declared
    local variable, and this is a write to that variable.

\end{itemize}

When writing to several local variables at once, they are listed in
stack order (rightmost is top-of-stack). The three following
constructions thus have identical effect:
\begin{verbatim}
    # Declare and initialize three variables.
    ->{a b c}

    # Declare three variables, then write to all of them at once.
    {a b c} ->[a b c]

    # Declare three variables, then write to them one at a time.
    {a b c} ->c ->b ->a
\end{verbatim}

In the third one, note that ``\verb|->a|'' is interpreted as the
accessor word that writes to the variable ``\verb|a|'', while
``\verb|-> a|'' would be parsed as the ``\verb|->|'' immediate function,
that then parses the token ``\verb|a|'', and adds to the current
function the effect of writing to ``\verb|a|'' (a \textsc{putlocal}
opcode), i.e. the same final outcome.

\subsection{Namespaces and Imports}\label{syntax:namespace}

At any point when processing source code, there is a \emph{current
namespace} which is used to qualify raw names for which no active alias
was found. The default current namespace is ``\verb|def|''.

The \emph{current aliases} are a mapping from raw names to qualified
names. Such mappings are built one at a time, and with \emph{import
lists}. An import list is made of all names defined in a given namespace
and declared ``exported''.

The ``\verb|std::namespace|'' immediate function changes the current
namespace:
\begin{verbatim}
    # Switch the current namespace to "foo"
    namespace foo
\end{verbatim}
The \verb|namespace| function parses the next token, which must be an
unqualified name.

When the new current namespace is changed, all the currently defined
aliases are cleared, and the import list for namespace \verb|std| is
loaded.

Aliases are defined with the ``\verb|std::alias|'' immediate function.
This function parses the next token:
\begin{itemize}

    \item If the next token is a qualified name \verb|n::r|, then the
    alias is for the raw name \verb|r| to the qualified name \verb|n::r|.

    \item Otherwise, the next token must be a raw name \verb|r|. The
    token that follows must then be a qualified name \verb|n::s|, and
    the mapping will be from \verb|r| to \verb|n::s|.

\end{itemize}

Import lists are obtained with the ``\verb|std::import|'' immediate
function. This function parses the next token, which must be an
unqualified name. That name is taken to be that of a namespace, and the
contents of the current list of exported names from that namespace are
added to the current list of aliases. Take care that the list of
imported names is the one at the time the \verb|import| clause is
processed; names later added to the import list of that namespace are
not automatically imported.

Name collisions are handled with the following rules:
\begin{itemize}

    \item A defined alias consists in the following:
    \begin{itemize}

        \item A \emph{source name}: this is the name \emph{for which}
        the alias is defined. It is always a raw name.

        \item A \emph{destination name}: this is the name \emph{to
        which} the alias is set. This name is an arbitrary string, but
        is usually a qualified name. It may also be the special
        \emph{invalid-name} value, which is distinct from all strings.

        \item A \emph{provenance flag}: it is meant to be set for names
        that have been set explicitly with \verb|std::alias|, and cleared
        otherwise.

    \end{itemize}

    \item When an \verb|std::alias| clause is used to define an alias
    for raw name \verb|r|:
    \begin{itemize}

        \item If there is no currently defined alias for \verb|r|, then
        the alias is defined as specified by the clause; its provenance
        flag is set.

        \item Otherwise, if there is a currently defined alias for
        \verb|r|, whose provenance flag is cleared, then the alias's
        destination is set to the destination name provided by the
        \verb|std::alias| clause, and its provenance flag is set.

        \item Otherwise, if the currently defined alias for \verb|r|,
        with its provenance flag set, has the same destination name as
        the one specified by the \verb|std::alias| clause, then nothing
        happens.

        \item Otherwise, the new alias points to a name distinct from
        the destination of the old alias, and the old alias has its
        provenance flag set: in that situation, an error is raised.

    \end{itemize}

    \item When an \verb|std::import| clause is used to load an import
    list, and the import list contains an alias for a raw name \verb|r|:
    \begin{itemize}

        \item If there is no currently defined alias for \verb|r|, then
        the alias is defined as specified by the clause; its provenance
        flag is cleared.

        \item Otherwise, if there is a currently defined alias for
        \verb|r| that points to the same name as the name defined in
        the import list, then nothing happens.

        \item Otherwise, if the currently defined alias for \verb|r|
        has its provenance flag set, then nothing happens.

        \item Otherwise, the new alias points to a name distinct from
        the destination of the old alias, and the old alias has its
        provenance flag cleared: in that situation, the alias's
        destination is set to \emph{invalid-name}.

    \end{itemize}

    \item Whenever an alias is \emph{used} for raw name \verb|r| (i.e.
    the raw name \verb|r| was encountered in a syntactic construction,
    and it is to be transformed thanks to the current aliases), and
    there is a currently defined alias for \verb|r| whose destination
    is \emph{invalid-name}, then an error is raised.

    \item The provenance flag has no influence on alias usage.

\end{itemize}

\begin{rationale}

Import lists are roughly similar to Java's whole-package imports, e.g.
``\verb|import java.util.*;|''. They have the same convenience of
getting easy access to many names with a single programming clause, but
they also share the same compatibility risks: if an import list is later
modified by the source package to include more exported names, these new
names may enter in conflict with other names defined by the source code
or imported from other lists. The mechanism with \emph{invalid-name} and
the provenance flag is meant to solve such issues along the following
principles:
\begin{itemize}

    \item The developer is supposed to know what happens in her own
    namespace. Thus, a conflict between two explicitly defined aliases
    is a programming error, hence sanctioned immediately.

    \item Similarly, a collision between an imported alias and an
    explicitly defined alias is resolved in favour of the latter: an
    explicit alias has precedence over an imported alias.

    \item A collision between two imported aliases is not resolved,
    but does not trigger an immediate exception: the import lists are
    considered to be out of reach of the developer, and thus may
    incur collisions that she cannot prevent. However, use of the name
    on which the collision occurs becomes ambiguous, and thus triggers
    an exception.

    \item Redefining an alias identically is always permitted.

    \item Order of declaration should not matter.

\end{itemize}

Ambiguous names (from collisions between import lists) can be resolved
with an explicit \verb|std::alias| clause, that will take precedence
over both import lists.

It is a matter of programming style whether to use import lists or
explicit aliases. The import list from \verb|std| is always loaded
because it would be very inconvenient to write code without (in that
respect, it is similar to OCaml's ``Pervasives'' module). Functions
from other namespaces may be used with explicit namespace names, or
explicit aliases, or import lists, or any combination thereof.

\end{rationale}

Existing syntax favours the declaration of ``simple aliases'' that map a
raw name to a qualified version of the same name; this is what the
\verb|<export>| keyword does when defining a function. Nevertheless,
other (to be defined) API may be used to make aliases that map raw names
to arbitrary strings, both explicitly and through import lists.

\subsection{Errors}

In all of the previous text, the expression ``raising an error'' was used
many times. An error terminates execution immediately and is not
recoverable. It may include an error code for reporting purposes.

\begin{rationale}
There are several models for error handling, notably the following:
\begin{itemize}

    \item Individual functions may report errors as special values, as
    in C: for instance, a \verb|read()| call on a file descriptor (on
    Unix-like systems) returns either the number of bytes that have been
    read, or the special value \verb|-1|.

    \item To avoid the need to put error codes in the same space as
    values, the result may be wrapped into a container that retains
    whether a value or an error was obtained, and additional syntactic
    constructions are provided to test for errors and obtain the
    result; this is how things are done in Rust.

    \item In languages where functions can return several values,
    functions may return the result \emph{and} an error code as separate
    values; this requires a special ``no error'' error code, that the
    caller can easily test, as well as a default value to return along
    with the error code in case of error. Go uses this mechanism.

    \item Errors may be reported through thrown exceptions, as in Java
    or C\#. Activation contexts are unwinded until a catch mechanism
    is reached.

\end{itemize}

All of these mechanisms are imperfect, in particular on small,
constrained systems. Error values require extra code to receive them,
test for them, and, more often that not, propagate error codes up the
call chain. Exceptions tend to allow for a more compact and efficient
implementation, in that they keep error handling out of the main
processing; however, catching exceptions implies more complicated
semantics that make code generation harder, and can increase code
footprint.

In T1, a cruder but simpler mechanism is used: any error terminates the
whole program, or, more accurately, the whole \emph{module}. One of the
points of T1 is to allow compact, efficient coroutines; thus, an
application that uses T1 is supposed to be split into several modules
that act as coroutines to each other. Each module has its own stack and
heap, and modules communicate with each other only through serialized
messages. For instance, in an SSL/TLS library, a module written in T1
could handle X.509 certificate validation; it receives the encoded
certificate chain, and returns the validation result (notably the public
key from the certificate). Any validation failure then cancels the
complete module, but not the application. In effect, T1 error management
is about concentrating handling at module boundaries. This also maps to
a clustering structure in which T1 modules might run on a distributed
system.

\end{rationale}

\newpage
\section{Compilation}

\emph{Compilation} is a step which optionally occurs at the end of
interpretation, when T1 is invoked ``as a compiler''; it can also be
triggered explicitly by the source code itself. Compilation takes as
input a list of \emph{entry points} (specific functions), and produces
an executable form of these functions and their transitive dependencies.
This process is meant to fulfill the following:
\begin{itemize}

    \item Compiled code is small and self-reliant. It can be invoked and
    run without requiring access to a bulky runtime system.

    \item Interpretation features, such as defining types or new
    functions, are not available in compiled code. Notably, compiled
    code cannot access type or function names.

    \item Compiled output should be amenable to integration within
    applications written in other languages, in particular C.

    \item The compiler offers strong guarantees on the usage of memory
    resources by compiled code: maximum data stack depth and maximum
    storage area for activation contexts (including local variables
    and locally allocated instances) are computed; and dynamic
    memory allocation, if supported at all, can be made to occur only
    in a specific, limited area provided by the caller that invokes
    the compiled code.

    \item Compiled code is proven not to trigger any error related
    to function invocation: whenever a function is invoked, there is
    exactly one matching function that is more precise than all other
    matching functions; and accessor functions called on instances that
    extend the structure on which the accessor was defined find an
    unambiguous instance on which the access is to be performed.

    \item Similarly, compiled code is proven never to read an
    uninitialized local variable, to let a reference to a locally
    allocated instance escape its activation context, or to attempt to
    write into a statically allocated instance.

\end{itemize}

Compilation can work only on a subset of valid codes; notable among the
restrictions is that compiled code cannot be generally recursive, since
such recursion would prevent computing strong bounds on stack depth.

\begin{rationale}

Banning recursion is controversial, especially since most functional
languages instead strive to use recursion to express most of flow
control. The two main reasons to forbid recursion in T1 are the
following:
\begin{itemize}

    \item Not allowing recursion means that the call tree is finished,
    which permits the general flow analysis (described below) to
    terminate.

    \item Recursion allocates memory in spaces which are scarce on
    memory resources. T1 aims at being useful for small embedded systems
    that have only a few kilobytes of RAM in total; however, even on
    bigger systems, stacks are small. For instance, a typical modern
    desktop system or laptop will have gigabytes of RAM, but the stack
    allocated for a thread is smaller (8 megabytes by default on Linux).
    Common sense dictates that if unbounded memory allocation occurs, it
    should not be done in an area which is a thousand times smaller than
    the heap, and for which the only detection mechanism for allocation
    failure is \verb|SIGSEGV|.

\end{itemize}

In a future version, \emph{tail calls} may be implemented, and tail
recursion allowed. In a tail call, the activation context of the caller
is released first, and when the callee returns, control is passed back
not to the caller, but the caller's caller. If a tail call does not
imply undue stack growth, then it won't prevent computing finite bounds
on stack depth, and it \emph{should} be manageable by flow analysis.

\end{rationale}

\subsection{Flow Analysis And Types}

\emph{Flow analysis} is the central step of compilation. Consider
the following code excerpt:

\begin{verbatim}
    : triple (object)
        dup dup + + ;
    : main ()
        4i32 triple println
        "foo" triple println ;
\end{verbatim}

The flow analysis starts with the entry point (\verb|main|) and an empty
stack. Then, after the \verb|4i32| token, the stack should contain
exactly one element of type \verb|i32|. At that point, the \verb|triple|
function is invoked. There is exactly one matching function of that name
for a stack with one element of type \verb|i32|, and therefore the call
is unambiguous.

Analysis proceeds with the \verb|triple| function. Crucially, analysis
of \verb|main| is not finished; it will be continued when this call to
\verb|triple| is done. Within \verb|triple|, the calls to \verb|dup| and
\verb|+| are followed; notably, when the first \verb|+| call is reached,
the stack is determined to contain three elements of type \verb|i32|.
When the end of \verb|triple| is reached, the stack is back to
containing one element of type \verb|i32|. At that point, flow analysis
jumps back to the caller (\verb|main|) and can proceed to the next call
(\verb|println|) since it is now known that this call is potentially
reachable (the \verb|triple| function may return) and also which stack
contents to expect at that time. The call to \verb|println| is resolved
to the function of that name that expects an element of type \verb|i32|.

Later on, the analysis reaches the second call to \verb|triple| in the
\verb|main| function. For that one, the stack contains one element of
type \verb|string|\footnote{Strictly speaking, \texttt{\textbf{string}}
is merely an alias on \texttt{\textbf{(std::u8 std::array)}}, but we
will use the name \texttt{\textbf{string}} for the clarity of the
exposition.}. There is still one matching function of name
\verb|triple|, and this is the same one as previously (indeed, there's
only one \verb|triple| function defined in this example, so only that
one may be called). However, the flow analysis of \verb|triple| will be
done \emph{again}: everything is done as if that call was a new one.

In that new call to \verb|triple|, the stack initially contains one
\verb|string|; after the two \verb|dup| calls, it contains three
\verb|string| elements; then, the \verb|+| calls will be resolved to the
function that ``adds'' strings (it concatenates them into newly
heap-allocated string values). That second analysis of \verb|triple|
concludes and returns a single \verb|string|. In \verb|main|, the second
\verb|println| call is resolved to the function of that name that
expects one element of type \verb|string| (not the same one as the one
that expects an \verb|i32|).

The salient points of this process are the following:
\begin{itemize}

    \item The \verb|triple| function has been \emph{registered} with one
    parameter type, which is generic (\verb|object| matches all value
    types). It cannot really be called on values of every type; for
    instance, it cannot be called on \verb|bool| since there is no
    defined \verb|+| function that works on \verb|bool| values. But it
    does not matter that the function could \emph{in abstracto} be
    invoked on values on which it would not work; what counts is whether
    such an invalid call is actually made in the program at hand. The
    flow analysis determines that all calls to \verb|triple| will work,
    and that is sufficient.

    \item Similarly, \verb|triple| could have been registered with no
    parameter at all (``\verb|: triple ()|''). During flow analysis, the
    compiler would still have known that at the time the function is
    invoked, there is a value on the stack, and the first \verb|dup|
    call won't underflow. Types for function registration are used
    \emph{only} to determine which function is called, not to restrict
    the actual usage of values on the stack\footnote{It is still good
    software engineering to register functions with exactly the
    parameters that it is going to use, if only for better source code
    readability.}.

    \item The fact that each \verb|triple| call has its own analysis
    avoids type merging trouble. If both calls were the same node in the
    call graph, then the flow analysis would be faced with calling
    \verb|+| on a stack with three elements, each being either a
    \verb|string| or an \verb|i32|. Such a call would not succeed
    because there is no \verb|+| function that can work over a
    \verb|string| and an \verb|i32|; the compiler would reject the code
    as making a call which is potentially unsolvable. In this case,
    duplicating the \verb|triple| node allows the flow analysis to keep
    track of the fact that while all three stack elements at this point
    may be of type \verb|i32| or \verb|string|, they all three have the
    same type, and cross combinations are not possible.

\end{itemize}

The node duplication means that, as far as flow analysis is concerned,
the ``call graph'' is a \emph{call tree}.

\begin{rationale}
Duplication of nodes for function calls is what makes all function
``generic'' in the Java or C\# sense. But since the analysis is done
only deductively, i.e. based on what may be on the stack at that
point of the program, there is no need for a syntax to express what
type combinations are allowed. Again, T1 does not care whether a given
function could work on all input values that may exist in the universe,
only that it would work with what may actually be present on the stack
at the time of the call.
\end{rationale}

\emph{Type merging} may still occur because of jump opcodes. For
instance, consider this function:

\begin{verbatim}
    : muxprint (object object bool)
        if drop else swap drop then println ;
\end{verbatim}

Suppose that the top three elements for a call to \verb|muxprint| are
values of type \verb|A|, \verb|B| and \verb|bool|. In the built function,
the call to \verb|println| can be reached from two points: this could
follow the ``\verb|swap drop|'' sequence (the boolean was \verb|false|,
the value of type \verb|A| has been dropped, the stack now contains an
object of type \verb|B|), or be reached through the jump that is implicit
in the \verb|else| construction. In the latter case, the top of the stack
will be a value of type \verb|A|.

Thus, flow analysis will consider that when \verb|println| is called,
the stack may contain one element which is of type \verb|A| or of type
\verb|B|. The call will be accepted only if it is solvable in both
cases. If the two cases are solvable, but lead to distinct functions,
then both functions will be analyzed, each with its own context.

For the purposes of flow analysis, all individual conditional jump
opcodes are considered independent of each other. This means that
the following cannot be compiled successfully:

\begin{verbatim}
    : foo (bool)
        ->{x} {y}
        x if 42 ->y then
        x if y println then ;
\end{verbatim}

Indeed, this function uses the provided input value (stored in the local
variable \verb|x|) to decide whether to put the integer \verb|42| in the
variable \verb|y| (first ``\verb|if|'' clause), and whether to print the
contents of the \verb|y| variable (second ``\verb|if|'' clause). The
compiler does not notice that both jumps use the same control value;
instead, it considers that the jumps are independant of each other, and,
in particular, the first jump may be taken, thus skipping the
initialization of \verb|y|, while the second would not, leading to the
read of the potentially uninitialized variable \verb|y|.

\begin{rationale}
The idea that conditional jumps are independent of each other has been
borrowed from Java. Indeed, with the equivalent Java code:
\begin{verbatim}
    static void foo(boolean x) {
        int y;
        if (x) {
            y = 42;
        }
        if (x) {
            System.out.println(y);
        }
    }
\end{verbatim}
Java compilation fails with the error ``variable \verb|y| might not
have been initialized''.

This is not considered a great restriction. In practical Java
development, that kind of error occurs mostly when adding debug code to
an existing function, activated with a global ``debug'' flag.
\end{rationale}

The notion of \emph{type} used for the flow analysis is the combination
of the \verb|std::type| for the value, and the \emph{object allocation
point}. An object allocation point is one of the following:
\begin{itemize}

    \item static allocation (conceptually, in ROM, thus non-modifiable);

    \item the heap;

    \item a specific local slot within the activation context of a
    specific function call, i.e. a node in the call tree.

\end{itemize}
This information is the basis for the escape analysis (making sure that
instances allocated in activation contexts are not reachable after the
called function has returned) and for the verification of constant
objects (static allocation corresponds to \verb|const| definitions in C,
and thus normally end up in non-modifiable memory).

Apart from the stack contents, the types of local variables at any point
in a given function (for a given activation context, i.e. within a node
in the call tree), is also maintained. A special \verb|nil| type is used
for uninitialized local variables; any attempt at reading \verb|nil| is
rejected at compilation time.

Types of values written in object fields are tracked. The container
types are distinguished by allocation point, but all instances with the
same allocation point use the same tracking. Therefore, writing a value
of type \verb|int| in the field \verb|x| of a heap-allocated object of
type \verb|T| implies that, from the point of view of the flow analyzer,
all objects of type \verb|T| that are heap-allocated may contain, at all
times, a value of type \verb|int| in their \verb|x| field.

Note that any merging may enrich the list of possible types in a stack
slot, local variable or object field, and trigger further flow analysis
for all parts of the call tree that depend on it.

\subsection{Constraints}

The following constraints are enforced by the flow analyzer; any
violation implies a compilation failure:
\begin{itemize}

    \item The call tree must be finished.

    \item At every merge point, the stack depth is the same for all
    code paths leading to that point.

    \item No merge between basic types (booleans and small modular
    integers), or between a basic type and a non-basic type, may occur,
    whether on the stack, in local variables, or within fields of
    a structure type.

    \item When a local variable is read, it may not contain \verb|nil|
    (which marks the uninitialized state).

    \item No write to a field of an object with static allocation may
    happen.

    \item Whenever a value has an allocation point tied to a given node
    $N_1$ in the call tree, and it is written in a field of an other
    object, then that other object must have an allocation point tied
    to a node $N_2$, and $N_2$ must be either equal to $N_1$, or a
    descendant of $N_2$ in the call tree.

    \item When a function returns, the stack contents must not contain
    any value whose allocation point is the node of that function in the
    call tree.

    \item For every function call, all possible combinations of types on
    the stack at that point must be solvable, i.e. lead to a single most
    precise function call.

    \item When a field accessor is invoked, the field must be
    unambiguously located for all possible types of the owner object.

\end{itemize}

Note that some special functions do not return (e.g. \verb|std::fail|).
This is detected during flow analysis. As such, some opcodes may be
unreachable; these will be trimmed during code generation.

\begin{rationale}
Each basic type may be merged only with itself because basic types have
specific storage requirements, that may differ from those of ``normal''
values (which are pointers). In the generated code, values of basic
types may be passed around on a different, dedicated stack, or different
registers. Similarly, an object field declared with type \verb|std::u8|
should correspond to a one-byte slot in the memory layout; a feature
of T1 is that object layouts are predictable, so that they can be
accessed from C code.

The finiteness of the call tree is enforced with nested call counters:
when a node is entered that corresponds to a given function $f$, the
counter for $f$ is incremented; it is decremented when leaving $f$. If
the counter goes over a given threshold, then compilation stops with an
explicit message. This necessarily detects all infinite trees, since
there are only a finite number of functions, each with a finite number
of opcodes: an infinite tree can be obtained only through infinite
recursion.

Some finite recursion is still tolerated. This allows for some cases
where the same generic function is used for several levels of a nested
structure, but with distinct types that guarantee against unbounded
recursion.
\end{rationale}

\subsection{Code Generation}

Code generation occurs after flow analysis has completed successfully.
How code is generated depends on the target type; the compiler may
produce portable threaded code, or native code, or WASM, or anything
else. Generated code includes all functions that are part of the
call tree; other functions are automatically excluded.

A generic \emph{function merging} process occurs during code generation.
In the flow analysis, functions were duplicated: the same piece of code
may yield several distinct nodes in the call tree. When generating code,
these nodes may be merged back. This is subject to some restrictions and
subtleties:
\begin{itemize}

    \item Some merge operations may not be feasible. In our example with
    the \verb|triple| function, one of the nodes works on \verb|i32|
    values while the other uses \verb|string| values. In generated code,
    these values use different storage techniques (e.g. stack slots of
    different size, or different registers), which may preclude merging.

    \item Even when function merging is possible, it may be undesirable
    for performance: for instance, the unmerged function may have only
    simple calls (each \verb|triple| function node calls a single
    well-defined \verb|+| function), while the merged function may need
    a type-based dynamic dispatch (if the \verb|i32| and \verb|string|
    could be merged, the \verb|triple| function would have to look at
    the runtime type of the values to decide which \verb|+| version to
    call).

\end{itemize}

\begin{rationale}
In general, in T1, we aim at code compacity, hence apply merging
whenever possible. A future annotation will allow to explicitly tag some
functions as prohibiting non-trivial merging, i.e. when the relevant
types are not all strictly identical. This would reproduce the trade-offs
usually seen in C with \verb|inline| functions.
\end{rationale}


\end{document}
